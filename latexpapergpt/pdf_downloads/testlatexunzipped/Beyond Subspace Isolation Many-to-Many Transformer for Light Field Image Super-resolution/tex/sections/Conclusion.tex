\section{Conclusion}
In this paper, we have revealed the prevalent challenge of subspace isolation caused by the One-to-One scheme and present the novel concept of Many-to-Many Transformers (M2MT) as a new scheme to address this issue. The proposed M2MT is empowered with complete access to all pixels across all SAIs in a LF image to capture comprehensive long-range correlation dependencies. With M2MT as a pivotal component, we have proposed a simple yet effective M2MT-Net for LFSR. Extensive experiments on various public datasets have demonstrated that M2MT-Net surpasses state-of-the-art methods to a considerable extent. Its superiority is evidenced by visual interpretability in our in-depth analysis using the LAM technique, which highlights that M2MT involves a substantially broader range of pixels across wider SAIs beyond subspace isolation, signifying its truly global context and a more comprehensive modeling of correlation dependencies.

Looking ahead, there are some promising directions for improving M2MT in future works. Firstly, extending M2MT to EPI subspaces could potentially reinforce M2MT's capacity to process LF images with large disparities like the \textit{STFgantry} dataset \cite{vaishSTFgantry_2008}, akin to EPIT \cite{liangEPIT_arXiv2023}. However, the unique characteristics of EPIs present distinct challenges that necessitate further investigation. Secondly, M2MT-Net currently carries a relatively substantial model size when compared to other Transformer-based methods, despite its commendable speed. This is primarily attributed to the correlation encoding and decoding processes within M2MT. It would be advantageous to explore lightweight alternatives to achieve a balanced trade-off between computational efficiency and performance in future research endeavors.