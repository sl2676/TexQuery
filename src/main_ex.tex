\documentclass[aps,prx,showpacs,floatfix,twocolumn,superscriptaddress,nofootinbib,longbibliography]{revtex4-2}
\usepackage{graphicx}
\usepackage{bm,amsmath}
\usepackage{dcolumn}
\usepackage{amsfonts,amssymb}
\usepackage[version=4]{mhchem}
\usepackage{diagbox}
\usepackage{hyperref}% add hypertext capabilities
\usepackage{subfigure}
\begin{document}

\title{Quantum phase transitions  and composite  excitations of antiferromagnetic quantum spin trimer chains in a magnetic field }

\author{Jun-Qing Cheng}

\affiliation{State Key Laboratory of Optoelectronic Materials and Technologies, Center for Neutron Science and Technology, Guangdong Provincial Key
Laboratory of Magnetoelectric Physics and Devices, School of Physics, Sun Yat-Sen University, Guangzhou 510275, China}
\affiliation{School of Physical Sciences, Great Bay University, 523000, Dongguan, China, and
Great Bay Institute for Advanced Study, Dongguan 523000, China}

\author{Zhi-Yao Ning}

\affiliation{State Key Laboratory of Optoelectronic Materials and Technologies, Center for Neutron Science and Technology, Guangdong Provincial Key
Laboratory of Magnetoelectric Physics and Devices, School of Physics, Sun Yat-Sen University, Guangzhou 510275, China}


\author{Han-Qing Wu}
\email{wuhanq3@mail.sysu.edu.cn}
\affiliation{State Key Laboratory of Optoelectronic Materials and Technologies, Center for Neutron Science and Technology, Guangdong Provincial Key
Laboratory of Magnetoelectric Physics and Devices, School of Physics, Sun Yat-Sen University, Guangzhou 510275, China}


\author{Dao-Xin Yao}
\email{yaodaox@mail.sysu.edu.cn}
\affiliation{State Key Laboratory of Optoelectronic Materials and Technologies, Center for Neutron Science and Technology, Guangdong Provincial Key
Laboratory of Magnetoelectric Physics and Devices, School of Physics, Sun Yat-Sen University, Guangzhou 510275, China}


\date{\today}

\begin{abstract}
	Motivated by recent advancements in theoretical and experimental studies on the high-energy excitations, we theoretically explore the quantum phase transitions and composite  excitations of the antiferromagnetic trimer chains in a magnetic field using the exact diagonalization, density matrix renormalization group, time-dependent variational principle and  cluster perturbation theory. We utilize the entanglement entropy to uncover   the phase diagram, encompassing  the XY-I, $1/3$ magnetization plateau, XY-II and ferromagnetic phases.  The critical  XY-I and XY-II phases are both described by the conformal field theory with the central charge $c \simeq 1$. We reveal the diverse features of spin dynamics  in various phases by  using the dynamical structure factor. In the weak intertrimer interaction regime, we identify the intermediate-energy and high-energy modes in the  XY-I and $1/3$ magnetization plateau phases as the internal trimer excitations,  corresponding to the propagation of doublon and quarton, respectively. Notably, the magnetic field splits the high-energy spectra into two branches labeled as the upper quarton and lower quarton. Furthermore, we also explore the spin dynamics of a trimerized  model  closely related to the quantum magnet \ce{Na_2Cu_3Ge_4O_12}, and discuss the possibility of the quarton Bose-Einstein condensation.  Our results can be verified in the  inelastic neutron scattering experiments  and provide deep insights for  exploring the high-energy exotic excitations.
\end{abstract}

\maketitle

\section{\label{sec:level1} Introduction}

Understanding  the  profound physical nature in the strongly correlated many-body systems is a  challenging and fascinate task in modern condensed-matter physics. The interplay of strong quantum fluctuations and interactions gives rise to  a variety of exotic phenomena, which   have attracted significant interest in the study of low dimensional correlated systems. Particularly, quasi one-dimensional (1D) magnetic materials derived from the physics of the Heisenberg antiferromagnetic chain (HAC) and its extensions have been extensively investigated \cite{Mikeska2004}.
The  dynamical behaviors, characterized by the excitations, play a crucial role in  understanding the magnetic structure of quantum materials, and can be studied both theoretically and experimentally. Notably, 
 the gapless two-spinon continuum \cite{karbach97}  has been observed through inelastic neutron scattering in quasi 1D material \ce{KCuF_3} \cite{tennant93,lake13} and   the
frustrated ferromagnetic spin-$1/2$ chain compound \ce{LiCuVO_4} \cite{PhysRevLett.104.237207}. The multi-spin excitations can be detected using the 
resonant inelastic X-ray scattering (RIXS) technique \cite{PhysRevLett.106.157205,RIXS2018NC} in the HAC material \ce{Sr_2CuO_3}. Furthermore, the high-energy string excitations have  been proposed as the dominant excitations in the isotropic Heisenberg antiferromagnet based on the Bethe ansatz \cite{PhysRevLett.102.037203}, and  recently  been observed in the antiferromagnetic Heisenberg–Ising chain compounds \ce{SrCo_2V_2O_8} and \ce{BaCo_2V_2O_8} under strong longitudinal magnetic fields using the high-resolution terahertz spectroscopy  \cite{wang2018experimental,PRL2019ZheWang}.


Besides the uniform spin chains, 
quantum materials often exhibit structures consisting of more than one spin per unit cell, resulting in a wider range of magnetic properties.
Ladder systems are well studied examples of 1D systems with unit cells, where the gapless or gapped  excitation spectrum  depend on the  rungs consisting of an odd or even number of $S=1/2$ spins, respectively \cite{dagotto96}. This behavior is  similar
to the Haldane's conjecture on spin chains with half-odd-integer or integer spins \cite{haldane1983}. Among the experimental realizations, the two-leg ladder
compound \ce{(C_7H_{10}N)_2CuBr_4} is noteworthy due to the excellent agreement between its experimental results and the dynamic spin structure factor from the model calculations \cite{schmidiger13}. 
The  unit cells can  effectively form  linear chains,  with different behaviors observed for systems with an even or odd number of spins in each cell. For example, the model with two-spin unit cells, with repeated couplings $J_1 - J_2$,
is well studied. A gap is present in the spectrum if $J_1 \not= J_2$, as the modulation leads to the spinons of the uniform the HAC ($J_1=J_2$) confining
into ``triplons'' that can be regarded as weakly bound spinon pairs if $J_2 \approx J_1$ \cite{doretto09}. However,
the model with three-spin unit cells, with repeated couplings $J_1 - J_1 - J_2$, has a gapless two-spinon continuum \cite{cheng2022}.  Furthermore,
the three-spin unit cells can also be coupled to each other in many other ways, as seen in the trimer chains like \ce{A_3Cu_3(PO_4)_4}
(\ce{A=Ca, Sr, Pb}) \cite{PhysRevB.71.144411,drillon19931d,belik2005long,PhysRevB.76.014409,PhysRevB.102.035137,PhysRevB.105.134423} and \ce{(C_5H_11NO_2)_2.3CuCl_2.2H_2O}
\cite{hasegawa2012magnetic}, where two spins in one trimer are coupled to two spins of their neighboring trimers. This is in contrast to the ladders, where all
spins in a rung are  uniformly coupled to spins in neighboring rungs. In the  iridate \ce{Ba_4Ir_3O_10}, where three-spin unit cells form the layered trimers, the anomalous fractional spinons  have been observed in the RIXS experiment \cite{PhysRevLett.129.207201}.


Recently, we have conducted an investigation into the spin dynamics of a trimer chain with repeated couplings
$J_1 - J_1 - J_2$ (intratrimer $J_1$ $\geq$ intertrimer $J_2$), and found that its low-energy effective model is the uniform HAC with spin $S=1/2$ in each trimer \cite{cheng2022}. The low-energy excitations are still the  gapless spinons in the reduced Brillouin zone, particularly for $J_2 \ll 1$.  
When $J_2 = J_1$, the trimer chain reduces to the conventional HAC with the two-spinon continuum, indicating that the low-energy spinons reside in different   Brillouin zones. 
Most interestingly, when $J_2/J_1 \ll 1$, the composite excitations of the
novel  quasi-particles, doublons and quartons have been predicted at the intermediate-energy and high-energy spectra, respectively, and immediately  confirmed in the 
inelastic neutron scattering  measurements on \ce{Na_2Cu_3Ge_4O_12} \cite{bera2022emergent}. As $J_2/J_1 \rightarrow 1$, the doublons and quartons lose their identity and fractionalize into the standard HAC spinon continuum. The high-energy spin excitations  exhibit intriguing physical properties, but remain insufficiently explored.   Previous studies have investigated these excitations
in the antiferromagnetic parent compounds of the high-Tc superconductors \cite{PhysRevLett.105.247001,Zhou2013,Ishii2014,PhysRevB.91.184513,Song2021}, and other systems described by the 2D Heisenberg models \cite{SAC2,dalla2015fractional,PhysRevB.108.224418,chang2023magnon}. It is crucial to delve deeper into these high-energy physics phenomena  that beyond the  spin waves, 
like the doublons, quartons and string excitations, including  the doublons, quartons and string excitations.


From a theoretical perspective, it is interesting to examine the influence  of a magnetic field on the doublons and quartons in the trimer chain illustrated in Fig.~\ref{model}. In this work, we employ various techniques including the  exact diagonalization (ED), density matrix renormalization group (DMRG) \cite{PhysRevLett.69.2863,PhysRevLett.93.076401,SCHOLLWOCK201196}, time-dependent variational principle (TDVP) \cite{PhysRevLett.107.070601, PhysRevB.94.165116},  and cluster perturbation theory (CPT) \cite{PhysRevB.48.418,PhysRevLett.84.522,RevModPhys.77.1027,PhysRevB.98.134410} to investigate the excitation spectra of trimer chain under the magnetic field. Firstly, we obtain the  phase paragram of ground state using the DMRG method. By mapping the entanglement entropy onto the parameter space,  we identify the  XY-I, $1/3$ magnetization plateau, XY-II and ferromagnetic phases. In the gapless XY-I and XY-II phases, both the central charges $c \simeq 1$ indicate that these two  phases   are well described by the  conformal field theory. Concerning the longitudinal excitations,  the incommensurate wave numbers of zero energy are dependent of the magnetization. Additionally,  we  study the intermediate-energy and high-energy excitations for small $g$  in the XY-I and $1/3$ magnetization plateau  phases. Our analysis demonstrates that  the intermediate-energy and high-energy modes are primarily governed by the internal trimer excitations,  referred to as the doublons and quartons, respectively. Furthermore, these features of excitation spectra can also be observed in the spin chain with trimer structure that is closely associated with the quantum magnet \ce{Na_2Cu_3Ge_4O_12 } \cite{bera2022emergent}. The magnetic field tuns the lower quarton to approach zero energy, indicating the magnetic-field-induced  quarton 
Bose-Einstein condensation (BEC).
Our results may facilitate further exploration of the high-energy spin excitation mechanisms in other systems containing clusters with odd spins.



   

%%%%%%%%%%%%%%%%%%%%%%%%%%%%%%%%%%%%%%%%%%%%%%%%%%%%%%%%%%%%%%%%%%%%%%%

\section{Results}

\subsection{Model}

The Hamiltonian of the spin-$1/2$ antiferromagnetic trimer chain in the presence of a longitudinal magnetic field with periodic boundary conditions reads
\begin{eqnarray}
	\mathcal{H}&=&\sum_{i=1}^{N}\left[ J_1 \left(\mathbf{S}_{i,a}\cdot \mathbf{S}_{i,b} +\mathbf{S}_{i,b} \cdot \mathbf{S}_{i,c} \right)
	+ J_2 \mathbf{S}_{i,c} \cdot\mathbf{S}_{i+1,a} \right] \nonumber \\ 
	&-& H_z \sum_{j=1}^{3N} S_j^z,
\end{eqnarray}
where $\mathbf{ S}_{i,\alpha}$ is the spin-$1/2$ operator at the $\gamma$-th site in the $i$-th trimer, the intratrimer labels $\gamma \in \{a,b,c\}$ are explained
by Fig.~\ref{model}. $H_z$ represents the strength of external magnetic field, which  breaks the $\mathrm{SU}(2)$ symmetry.  The system consists of a total of $N$ trimers, resulting in a system length of $L=3N$.  The tuning parameter $g$ is defined as $g = J_2/J_1$. For simplicity, we set the
intratrimer interaction $J_1=1$ as the energy unit, so that intertrimer interaction $J_2=g$. Our interest is in the range of 
coupling ratios $g \in (0,1)$ where the system evolves between the isolated trimers and the isotropic  HAC.



\subsection{\label{subsec:cells} Quantum phase transitions}

\begin{figure}[t]
\includegraphics[width=8cm]{model}
\caption{\textbf{Schematic representation of a trimer spin chain under a longitudinal magnetic field}. We here consider systems with $J_1 \geq J_2 >0$, and use the letters $a,b,c$ as indicated to
refer to the three spins within a unit cell.}
\label{model}
\end{figure} 


\begin{figure*}
\includegraphics[width=18cm]{phase-sz}
\caption{\label{diagram} \textbf{Quantum phase transitions.} (a) Phase diagram 
 obtained by employing the DMRG to map the entanglement entropy onto the parameter space  $(g,H_z)$  for a system with $L=180$ spins ($N=60$ trimers).  (b) Magnetization curves as a function of $H_z$ for different $g$ in a system  with $L=180$. Inset shows the width of $1/3$ magnetization plateau as function of $g$. (c) Entanglement entropy $S(L_A)$ as a function of the subsystem size $L_A$ with open boundary conditions.  Solid lines of inset are best fits to the CFT scaling form. The best-fit values for the 	central charges  are  given. (d)(e)(f)(g) Magnetization of every spin obtained by DMRG for four phases where $g=0.6$ and $H_z =0.3, 1.0, 1.5, 2.0$. }
\end{figure*}


In the absence of a magnetic field, the antiferromagnetic quantum spin trimer chain exhibits a gapless low-energy excitation known as the two-spinon continuum \cite{cheng2022}.  When a magnetic field is applied, the $\mathrm{SU}(2)$ symmetry is broken, leading to the emergence of a quantum phase transition driven by the competition between the interaction and magnetic field.  In this subsection, we aim to investigate the detailed phase diagram using  the  DMRG method. 
 
 
Quantum entanglement provides a unique perspective for unveiling the ground state properties of a many-body system, and   has been widely employed to study the quantum phase transitions \cite{RevModPhys.80.517,LAFLORENCIE20161,cheng2017,PhysRevLett.120.200602,PhysRevE.97.062134,PhysRevLett.128.020402}. Entanglement entropy, as a fundamental measurement of bipartite quantum  entanglement, can be readily obtained through DMRG calculations. Its definition is given by
\begin{eqnarray}
	S=-\mathrm{Tr}\left[\rho^\mathrm{A} \ln \rho^\mathrm{A}\right],
\end{eqnarray}
where the reduced density matrix $\rho^\mathrm{A}$ is the partial trace of the density matrix $\rho$ of whole system  $\rho^\mathrm{A}=Tr^\mathrm{B}\left[\rho\right]$. If $\mathrm{A}$ and $\mathrm{B}$ are entangled, the reduced density matrix must be a mixed state and the entanglement entropy quantifies this degree of mixing. By effectively analyzing the entanglement entropy, the characteristics of ground states in various quantum phases can be extracted. Therefore, the entanglement entropy serves as a viable and useful tool for investigating the quantum phase transitions.
As illustrated in Fig.~\ref{diagram}(a), the entanglement entropy reveals 
 four distinct phases  in the $(g,H_z)$ parameter space. Applying external magnetic field, a N\'{e}el to incommensurate  phase transition occurs,  leading the system into the XY-I Phase. However, the magnetic field is not strong enough to open a gap, thus the ground state is still gapless with  a nonzero entanglement entropy. In Fig.~\ref{diagram}(b), the magnetization grows  with the increase of magnetic field in this XY-I phase until a fractional magnetization plateau is present. 


\begin{figure*}[t]
	\includegraphics[width=14cm]{Szz-TDVP-g0.8}
   \caption{\label{szz-g8} \textbf{Dynamic spin structure factor $\mathcal{S}^{zz}  (q,\omega) $  obtained from DMRG-TDVP calculations for different phases.}  $\mathcal{S}^{zz}  (q,\omega) $ in  (a) XY-I phase, (b) $1/3 $ magnetization  plateau phase, and (c) XY-II phase. All results are from the case where $g=0.8$ and $L=120$. The  color coding of $\mathcal{S}^{zz} (q,\omega)$ uses a piecewise function with
   the boundary value $U_0=2$. 
   Below the boundary, the low-intensity portion is characterized by a linear mapping of the spectral
   function to the color bar, while above the boundary a logarithmic scale is used, $U=U_0+\log_{10}[\mathcal{S}^{zz}(q,\omega)]- \log_{10}(U_0)$.}
\end{figure*}

 The fractional magnetization plateau in the magnetization curves can be understood through  the Oshikawa-Yamanaka-Affleck (OYA) criterion \cite{PhysRevLett.78.1984}:
\begin{equation}
	n\left(s-m\right)=\mathrm{integer}  \label{OYA}
\end{equation}
where $n$ is the number of spin in a unit cell, $m$ is the magnetization per site in unit and $s$ denotes the magnitude of spin. In the trimer chain, it has $n=3$, $s=1/2$, so when $n(s-m)=0$  it has $m=1/2$ corresponding to the total polarized case, and when $n(s-m)=3$ it has $m=1/6$ corresponding to the $1/3$ magnetization plateau. Fig.~\ref{diagram}(b) clearly illustrates the presence of these two plateaus. In the $1/3$ magnetization plateau phase,   the external magnetic field is not strong enough to decouple the singlets.
As $g$ increases, the width of magnetization plateau  decreases, eventually disappearing  at $g=0$ where the trimer chain becomes the uniform Heisenberg chain.
From the magnetization of each spin, as shown in Figs.~\ref{diagram}(d)(e)(f)(g), we can observe that the magnetization  follows a periodic pattern in terms of the trimerized structure. Specifically, the magnetization of central spin of each trimer  flips up as $H_z$ increases. Importantly, 
 the magnetization of  each spin remains fixed even as  the magnetic field increases  in the $1/3$ magnetization plateau phase. The ground state of an isolated trimer in the presence of magnetic field is given by, 
\begin{equation}
\left|0\right\rangle  = \frac{1}{{\sqrt 6 }}\left( {\left| { \uparrow  \uparrow  \downarrow } \right\rangle  - 2\left| { \uparrow  \downarrow  \uparrow } \right\rangle  + \left| { \downarrow  \uparrow  \uparrow } \right\rangle } \right),
\end{equation}
which is also the  antiferromagnetic trimer state of the Haldane plateau in one-dimensional $ (S,s)=(1,1/2)$ mixed spin chain where the spin $S=1$ is represented by two spins $S=1/2$ \cite{PhysRevB.65.214403}.  In the $1/3$ magnetization plateau phase, as shown in Fig.~\ref{diagram}(f), the expectation values of the $z$ component of three spins $a,b,c$ are  $0.322$, $-0.144$, and $0.322$, respectively, which are approximately coincide  with the ideal state with expectation values $1/3$, $-1/6$, and $1/3$.  It indicates that the $1/3$ magnetization plateau state exhibits the N\'{e}el order along the magnetic field, with each trimer having an effective spin $1/2$, giving the appearance of polarization for each trimer. 
 



As the magnetic field increases, the singlets are destroyed, leading to the presence of the XY-II phase.  In this phase, the system remains gapless with a nonzero entanglement entropy in ground state, and the average of magnetization is larger than the one in XY-I phase, and also increases with the magnetic field.
  As long as the magnetic field is strong enough, all spins become polarized, resulting in the formation of the other magnetization plateau and the presence of ferromagnetic phase. We also use entanglement properties to indicate the possible conformal field theory description of the gapless phases. 
  We consider a one-dimensional system of length $L$ and
divide it into two segments of length $L_\mathrm{A}$ and $L_\mathrm{B} =L - L_\mathrm{A}$. The density matrix of ground state for the entire system is denoted as  $\rho=\left|\psi\right\rangle \left\langle \psi \right|$, and the reduced density matrix can be obtained by tracing
over the degrees of freedom in $L_\mathrm{B}$, denoted as $\rho_\mathrm{A} = Tr_\mathrm{B} (\rho)$. The R\'enyi entanglement  entropy is  given by
\begin{equation}
S_\nu \left(\rho_\mathrm{A}\right)=\frac{1}{1-\nu} \ln \left(\operatorname{Tr}\left\{\rho_\mathrm{A}^\nu\right\}\right).
\end{equation}
 In the limit $ \nu \rightarrow 1$, the above expression 
reduces to the von Neumann entanglement entropy,
\begin{equation}
S(\rho_\mathrm{A})=\lim _{\nu \rightarrow 1} S_\nu \left(\rho_\mathrm{A}\right)=-\operatorname{Tr}\left[\rho_\mathrm{A} \ln \rho_\mathrm{A}\right].
\end{equation}
The R\'enyi entanglement  entropy of subsystem  $A$ follows the scaling form \cite{PhysRevLett.104.095701,PhysRevB.92.054411,PhysRevB.105.014435}:
\begin{equation}
S_\nu\left(L_\mathrm{A}\right)=S_\nu^{\log }\left(L_\mathrm{A}\right)+S_\nu^{\operatorname{osc}}\left(L_\mathrm{A}\right)+\tilde{c}_\nu,
\end{equation}
where  
\begin{equation}
S_\nu^{\log }\left(L_\mathrm{A}\right)=\frac{c}{6 \eta}\left(1+\frac{1}{\nu}\right) \ln \left\{\left[\frac{\eta L}{\pi} \sin \left(\frac{\pi L_\mathrm{A}}{L}\right)\right]\right\},
\end{equation}
and 
\begin{equation}
S_\nu^{\mathrm{osc}}\left(L_\mathrm{A}\right)=F_\nu\left(\frac{L_\mathrm{A}}{L}\right)\frac{\cos \left(2 k_F L_\mathrm{A}\right)}{\left|\frac{2 \eta L}{\pi} \sin \left(\pi L_\mathrm{A} / L\right)\right|^{\frac{2 \Delta_1}{\eta \nu}}}.
\end{equation}
Here, $\eta=1,2$ is for periodic and open boundary conditions, respectively. The  central charge
$c$, the Fermi momentum $k_F$, and the scaling dimension $\Delta_1$ are universal parameters. $F_{\nu} (L_\mathrm{A}/L) $ is a universal scaling function and $\tilde{c}$ is a nonuniversal constant. By fitting the DMRG data with these functions for $\nu =1$,   we extract the central charges of two XY phases as indicators of their universality classes. As shown in Fig.~\ref{diagram}(c), the two XY phases are both described by the conformal field theory with the central charges $c\simeq 1$. 





\begin{figure*}[t]
	\includegraphics[width=18cm]{Sxx-g0.8}
	\caption{\label{sxx-g8} \textbf{$\mathcal{S}^{xx}  (q,\omega) $  obtained from CPT and DMRG-TDVP calculations for different phases.}  $\mathcal{S}^{xx}  (q,\omega) $ in  (a)(e) XY-I phase, (b)(f) $1/3 $ magnetization  plateau phase, (c)(g) XY-II phase, and (d)(h) Ferromagnetic phase. All results are from the case where $g=0.8$, and the DMRG-TDVP calculations are from the length $L=120$. The  color coding of $\mathcal{S}^{xx} (q,\omega)$ uses a piecewise function with the boundary value $U_0=2$, which can be also found in the caption of Fig.~\ref{szz-g8}. }
\end{figure*}



\subsection{Excitation spectra}

In this section, we present the spin excitation spectra of the trimer chain in external magnetic field using the dynamical structure factor (DSF):
\begin{equation}
\mathcal{S}^{\alpha \beta}(q, \omega)=\sum_j \mathrm{e}^{-\mathrm{i} q j}\left[\int_{-\infty}^{\infty} \mathrm{d} t \mathrm{e}^{\mathrm{i} \omega t}\left\langle\hat{S}_j^\alpha(t) \hat{S}_0^\beta\right\rangle\right],
\end{equation}
where $\alpha,\beta$ refer to spin components $x$, $y$, and $z$. We calculated the DSF using the CPT and DMRG-TDVP to study the spin dynamics under the modification of control parameters, $g$ and $H_z$. The calculation details can be found in the Sec.~\ref{method}.


Let us first consider the longitudinal excitation spectrum $\mathcal{S}^{zz}  (q,\omega) $ at  $g=0.8$ for XY-I phase, $1/3$ magnetic plateau phase and XY-II phase. As shown in Fig.~\ref{szz-g8},  the excitations are gapless in  two XY phases and gapped in the $1/3$ magnetic plateau phase. For the ferromagnetic phase, all spins are polarized in the $z$ direction, the longitudinal excitation spectrum in this phase is absent. To understand the spin dynamics of XY-I and XY-II phases, let us exmine the zero-energy excitations. 
The incommensurability appearing in the spin dynamics for an AF spin-$1/2$ chain under a longitudinal magnetic field can be understood in terms of spinless fermion language \cite{takayoshi2023phase}. The longitudinal magnetic field is equivalent to a chemical potential, which modifies the filling of the band and splits the degeneracy of the electron-hole bands. The intraband and interband zero-energy excitations  corresponds to the longitudinal and transverse fluctuations, respectively. 
For the trimer chain with a longitudinal magnetic field, we can also observe the  incommensurability arises from the splitting of the bands. In the XY-I phase, see Fig.~\ref{szz-g8}(a), the 
longitudinal excitations conserve  the total number of particles without changing the magnetization of ground state, which indicates that the 
incommensurate fluctuations reaching zero energy at $q=(1\pm m_z)\pi$, where $m_z=0.1111$ is the 
magnetization normalized by its saturation value. In the XY-II phase, see Fig.~\ref{szz-g8}(c), the 
incommensurate fluctuations reach the zero energy at $q=(1\pm m_z)\pi$ where $m_z = 0.5697$. 
In Supplementary Note 1, we present the longitudinal excitations obtained from our extensive ED calculations. In addition,  the incommensurabilities observed in the spin dynamics are also consistent with the DMRG-TDVP results.

In Fig.~\ref{sxx-g8}, the transverse excitation spectra  $\mathcal{S}^{xx}  (q,\omega) $ for four phases are present. Figs.~\ref{sxx-g8}(a)(e) displays the transverse excitations $S^{xx}(q, \omega)$ changing the number of particles, which give rise to fluctuations reach the zero energy at incommensurate wave numbers $q=m_z \pi$ and $q=(2-m_z)\pi$ in addition to $q=\pi$. The spectral weight of $\mathcal{S}^{xx}  (q,\omega) $  is concentrated at the commensurate positions corresponding to each reciprocal lattice points of $q=\pi$ in the XY-I and XY-II phases.  At the high-energy regime, the continuum is observed both in the $1/3$ magnetization plateau and XY-II phases.  At the ferromagnetic phase $H_z=2.0$, see
Figs.~\ref{sxx-g8}(d)(h),  all spins are polarized,  the spin excitation is still the propagation of   magnon.  Some energy gaps are observed  at the   Brillouin zone  edges $q=\pi/3,2\pi/3,4\pi/3,5\pi/3 $ where the spin waves are diffracted due to the periodic potential of the trimerized interaction, therefore the magnons at the Brillouin zone edges  have two energies for the same wave vector.







  


\begin{figure*}[t]
	\includegraphics[width=18cm]{Sxx-g0.3}
	\caption{\label{Sxx} \textbf{ $\mathcal{S}^{xx}  (q,\omega) $  obtained from CPT and DMRG-TDVP calculations for different phases with weak intertrimer interaction.}  $\mathcal{S}^{xx}  (q,\omega) $ in  (a)(e) XY-I phase, (b)(f) $1/3 $ magnetization  plateau phase, (c)(g) XY-II phase, and (d)(h) Ferromagnetic phase. All results are from the case where $g=0.3$, and the DMRG-TDVP calculations are from the length $L=120$. The  color coding of $\mathcal{S}^{xx} (q,\omega)$ uses a piecewise function with the boundary value $U_0=2$. }
\end{figure*}

\begin{figure*}[t]
	\includegraphics[width=16cm]{energy}
	\caption{\label{energy} \textbf{ The level spectrum, wave functions and quantum numbers of one isolated trimer under a longitudinal magnetic field.} The second column lists the wave functions in the spin-$z$ basis, while the third column presents the spin structures using a basis of singlets (gray ovals and rounded shapes), zero-magnetization triplets (gray square shapes), and unpaired spins (arrows). The last column lists the total spin quantum number $S$, magnetic quantum number $M$, and the internal trimer excitations with $\Delta M = \pm 1$. }
\end{figure*}



\begin{figure*}[t]
	\includegraphics[width=18cm]{Sxx-plateau-dispersion}
	\caption{\label{Sxx-plateau-dispersion} \textbf{ $\mathcal{S}^{xx}  (q,\omega) $ in  $1/3 $ magnetization  plateau phase.} All results are obtained by  DMRG-TDVP calculations  for $L=120$, and the  color coding of $\mathcal{S}^{xx} (q,\omega)$ uses a piecewise function with the boundary value $U_0=2$. The dispersion lines with colors and numbers are corresponding to the different localized excitations in a single trimer. (1)(2)(4) are the excitations from $\left| 0 \right\rangle \rightarrow \left| 1 \right\rangle$, $\left| 0 \right\rangle \rightarrow \left| 3 \right\rangle$ and $\left| 0 \right\rangle \rightarrow \left| 6 \right\rangle$ with $\Delta M =-1$, respectively. (3) is the  excitations from $\left| 0 \right\rangle \rightarrow \left| 4 \right\rangle$ with $\Delta M =1$. }
\end{figure*}





In our previous study shown in Ref.~\cite{cheng2022}, we have found that the smaller $g$ induces  rich intermediate-energy and high-energy  excitations beyond the spin wave. Therefore, it is of great interest  to investigate the evolutions of intermediate-energy  and high-energy quasiparticles referred  to as the doublons and quartons under the magnetic field. Next, we focus on the weak intertrimer coupling  $g=0.3$  to study their dynamical evolutions. 
In this case,  the system is nearly isolated trimer, and  can be analyzed simply. From Figs.~\ref{Sxx}(a)(e), we observe that the low-energy excitation is similar to the excitation spectrum of conventional spinon under the magnetic field, a  split of the dispersion relation occurs, which is characterized by the emergent fermions of Heisenberg chain in the magnetic field \cite{PhysRevB.94.125130}. The intermediate-energy spectrum has little separation near $q=\pi/3$ and $q=5\pi/3$, and a continuum emerges possibly due to the propagation of doublons dressed by spinons \cite{cheng2022}. The high-energy spectrum is clearly split  into two branches by the magnetic field.





As $H_z $ increases, see Figs.~\ref{Sxx}(b)(f), the excitation gap  opens, and the system enters a $1/3 $ magnetization  plateau phase,  the lower   spectrum of high-energy excitation with $\Delta M=-1$ shifts to the low-energy regime. When $H_z =1.5$, the system enters the gapless XY-II phase (see Figs.~\ref{Sxx}(c)(g)), the low-energy, intermediate-energy and high-energy spectra are well separated. At the ferromagnetic phase $H_z=2.0$, shown in
Figs.~\ref{Sxx}(d)(h),    the spin excitation is still the spin wave since all spins are polarized.  Some energy gaps are also observed  at the   Brillouin zone  edges $q=\pi/3,2\pi/3,4\pi/3,5\pi/3 $, which is consistent with the results of Figs.~\ref{sxx-g8}(d)(h).



%%%%%%%%%%%%%%%%%%%%%%%%%%%%%%%%%%%%%%%%%%%%%%%%%%%%%%%%


\begin{figure*}[t]
	\includegraphics[width=18cm]{material}
	\caption{\textbf{Trimer model related to the experimental material \ce{Na_2Cu_3Ge_4O_12}.} (a) Schematic representation of trimer model with next-nearest neighbor intratrimer exchange couplings $J_3$, and $J_2=J_3=0.18J_1$. The spins $a$, $b$ and $c$ are the three \ce{Cu^2+} spins within a trimer unit. (b) Energy levels as functions of magnetic field $H_z$.
	(c) Eigenevergies, wave functions, and quantum numbers of an isolated trimer unit in the material under the magnetic field $H_z$. $\mathcal{S}^{xx}  (q,\omega) $ of experimental materials \ce{Na_2Cu_3Ge_4O_12 } in  (d) XY-I phase, (e)(f) $1/3 $ magnetization  plateau phase, (g) XY-II phase, and (h) Ferromagnetic phase obtained by  DMRG-TDVP for $L=120$. The  color coding of $\mathcal{S}^{xx} (q,\omega)$ uses a piecewise function with  the boundary value $U_0=2$.}
	\label{material}
	\end{figure*}


\subsection{Excitations mechanisms}

In order to gain a deeper understanding of  the intermediate-energy and high-energy spin dynamics, it is instructive to analyze the complete level spectrum and corresponding eigenvectors of a single trimer. As depicted in Fig.~\ref{energy}, the application of a magnetic field results in the splitting of three energy levels into eight levels.  Notably,
  the eigenvectors, spin quantum numbers and magnetic quantum numbers   remain invariant.  When $H_z \leq 1.5$, the ground state of one trimer is denoted as $\left|0\right\rangle$ with the energy $E_0 = -J_1-H_z/2$. 
Considering the excitations with $\left|\Delta M\right|=1$ from $\left|0\right\rangle$,  only four cases satisfy this condition as indicated  in the last column of Fig.~\ref{energy}.
For small $g$, the coupling between trimers can be treated as a perturbation of product state of isolated trimers, which has been confirmed in our previous study of trimer chain without the magnetic field \cite{cheng2022}. 
Here, the perburbative analysis is still an effective tool to handle with the spin excitations of trimer chain under the magnetic field, particularly in the XY-I and $1/3 $ magnetization  plateau phases.
In the XY-I phase with small $g$, a weak magnetic field induces an incommensurate ground state with  slight magnetization. By using the  ground state $\left|0\right\rangle$ and first excited state $\left|1\right\rangle$ of single trimer,  we can  construct an  approximate ground state with the antiferromagnetic order, such as $\left|\psi\right\rangle_g = \left|0101\dots01\right\rangle$. Consequently,  we are able to calculate the dispersion relations corresponding to the intermediate-energy and high-energy excitations with $\left|\Delta M \right|= 1$ by employing  only $N=4$ trimers, details can be found in the Supplementary Note 2. 
Regarding the intermediate-energy excitations, 
four  dispersion relations 	 appear  at $\omega \propto J_1$, as shown in Fig.~\ref{Sxx}(e), and describe the localized excitations  from $\left|0\right\rangle$ to $\left|3\right\rangle$,
\begin{equation}
\label{intermediate}
	\epsilon_{\mathrm{D}} (q)= \left\{
    \begin{split}
   &-  \frac{1}{3} g \cos{(3q)}+E_1 -E_0 + H_z - \frac{1}{9}g, 
    \\
    &- \frac{2}{9} g \cos{(3q)}+E_1 -E_0 + H_z, 
    \\
    &- \frac{2}{9} g \cos{(3q)}+E_1 -E_0 + H_z + \frac{2}{9}g, 
    \\
    &- \frac{2}{9} g \cos{(3q)}+E_1-E_0  + H_z + \frac{2}{9}g.
    \end{split}
\right.
\end{equation}
It can be found that these dispersion relations depend on the gap $E_1-E_0$ between the ground state and first excited state,  and raises with the increase of magnetic field $H_z$.
Comparing to the case $H_z=0$ \cite{cheng2022}, only one branch of doublons is left under the effect of magnetic field, thus the intermediate-energy excitation corresponds to the generation of doublons.
These dispersion lines are  not well coincide with the spectrum due to the approximation of  ground state. Additionally,  we observe a continuum that may originate  from the bound spinons.  The central doublet dressed these spinons  propagates through the system, resulting in various internal modes of these composite excitations, which in turn leads to a band of finite width in energy of these excitations. 


 For the high-energy excitations, the magnetic field splits the spectrum into two branches as shown in Figs.~\ref{Sxx}(a)(e).
 Both branches arise from the high-energy internal trimer excitations. We refer to the upper branch (excitation from  $\left|0\right\rangle$ to $\left|6\right\rangle$) as the  upper quarton, and the lower one (excitation from  $\left|0\right\rangle$ to $\left|4\right\rangle$) as the lower quarton.
The    dispersion relations for the  upper quarton are given by,
\begin{eqnarray}
	\label{high1}
\epsilon_{\mathrm{UQ}} (q)=\left\{
    \begin{split}
& \frac{1}{18} g \cos{(3q)}+E_2 -E_0 + H_z - \frac{1}{6}g,\\
& \frac{2}{9} g \cos{(3q)}+E_2 -E_0 + H_z - \frac{1}{6}g,\\
&\frac{2}{9} g \cos{(3q)}+E_2 -E_0 + H_z + \frac{1}{6}g,
\end{split}
\right.
\end{eqnarray}
and the ones for  lower quarton are given by,
\begin{eqnarray}
	\label{high2}
	\epsilon_{\mathrm{LQ}} (q)=\left\{
    \begin{split}
& \frac{1}{6} g \cos{(3q)}+E_2 -E_0 - H_z + \frac{1}{18}g,\\
& \frac{2}{9} g \cos{(3q)}+E_2 -E_0 - H_z + \frac{1}{18}g,\\
& \frac{2}{9} g \cos{(3q)}+E_2 -E_0 - H_z +\frac{1}{6}g.
\end{split}
\right.
\end{eqnarray}
 These dispersion relations demonstrate a strong agreement with the DMRG-TDVP results regarding the location of these excitations and their band widths, which suggests that the picture of localized excitation is correct,  despite  the calculation involves a very rough approximation of the ground state.
 Consequently, the high-energy  quartons  persist in this XY-I phase at small $g$.  


In the $1/3$ magnetization plateau phase, each trimer possesses  an effective magnetic quantum number $1/2$, resembling a polarized spin as a unit cell. We can construct the ground state of the $1/3$ magnetization plateau using the ground state of single trimer, $\left|\psi\right\rangle_g = \left|000\dots00\right\rangle$,  to study the spin dynamics.
The low-energy spin wave arises from the flip of one spin in the ferromagnetic state, which is well captured by the propagation of magnon. In this scenario, we flip an effective spin of one trimer, for example, replace one trimer from $\left|0\right\rangle$ to $\left|1\right\rangle$ in state $\left|\psi\right\rangle_g $, that results in the dispersion relation, 
\begin{eqnarray}
	\epsilon^{(1)}(q)=	\frac{4}{9} g \cos{(3q)}+ H_z - \frac{4}{9}g,
\end{eqnarray}
which coincides well with the low-energy excitation spectrum no matter how large the magnetic field is, see Fig.~\ref{Sxx-plateau-dispersion}. We refer to this excitation as the reduced spin wave  inspire of the conventional magnon picture. Moving on to the intermediate-energy excitations, where one trimer is excited from $\left|0\right\rangle$ to $\left|3\right\rangle$ with $\Delta M=1$, the corresponding  dispersion relation is 
\begin{eqnarray}
	&&\epsilon^{(2)}(q)=-\frac{1}{3} g \cos{(3q)}+ E_1-E_0+H_z-\frac{2}{9}g.
\end{eqnarray}
Here, the intermediate-energy mode is termed as the doublon rather than the magnon since it originates from the localized trimer excitation and possesses a higher gap than the low-energy magnon. 
For the high-energy excitations, two branches of excitation spectra arise, corresponding to the  excitation $\left|0\right\rangle \rightarrow \left|6\right\rangle$ with $\Delta M=-1$ and $\left|0\right\rangle \rightarrow \left|4\right\rangle$ with $\Delta M=1$. The respective  dispersion relations are given by,
\begin{eqnarray}
	&&\epsilon^{(3)}(q)=\frac{1}{6} g \cos{(3q)}+ E_2-E_0-H_z+\frac{1}{9}g,\\
	&&\epsilon^{(4)}(q)=\frac{1}{18} g \cos{(3q)} + E_2 -E_0+H_z-\frac{1}{3}g,
\end{eqnarray}
which are referred as  the high-energy  quartons.   Notably, 
it can be observed that  the reduced spin wave ($\left| 0 \right\rangle \rightarrow \left| 1 \right\rangle$), doublon ( $\left| 0 \right\rangle \rightarrow \left| 3 \right\rangle$) and upper quarton ( $\left| 0 \right\rangle \rightarrow \left| 6 \right\rangle$) share the same magnetization quantum number $\Delta M=-1$, and raises together as the magnetic field grows. Conversely,  the lower quarton  descends independently  due to its distinct magnetization quantum number $\Delta M=1$, ultimately becoming the low-energy spectrum when $H_z \geq 0.9$. More interestingly, from Figs.~\ref{Sxx}(c)(g), we can observe that even in the XY-II phase, the excitations with $\Delta M=-1$ are still present in the high-energy regime.



%%%%%%%%%%%%%%%%%%%%%%%%%%%%%%%%%%%%%%%%%%%%%%%%%%%%%%%%%%%%%%%%%%%%%%%%%%%%%%%
\section{Quantum  magnets}

It has been  discovered that  \ce{Na_2Cu_3Ge_4O_12 } serves as an excellent realization of the spin-$1/2$ Heisenberg antiferromagnetic trimer chain, where  \ce{Cu_3O_8 } constitutes the  trimers formed by three edge-sharing \ce{CuO4} square planes linearly. The magnetic \ce{Cu^2+} ions within
the \ce{CuO4} square planes exhibit quantum spin-$1/2$ \cite{JAP2014,bera2022emergent}.  Fig.~\ref{material}(a)  illustrates the simplified spin model, which includes an additional next-nearest neighbor intratrimer exchange coupling $J_3$. The Hamiltonian for this system is given by 
\begin{eqnarray}
	\mathcal{H}^\prime&=&\sum_{i=1}^{N} [ J_1 \left(\mathbf{S}_{i,a}\cdot \mathbf{S}_{i,b} +\mathbf{S}_{i,b} \cdot \mathbf{S}_{i,c} \right)
	+ J_2 \mathbf{S}_{i,c} \cdot\mathbf{S}_{i+1,a} \nonumber \\
	 &+& J_3 \mathbf{S}_{i,a} \cdot\mathbf{S}_{i,c}]
	-H_z \sum_{j=1}^{3N} S_j^z,
\end{eqnarray}
where the experimental measurements have determined the coupling strengths as $J_1 =235K$ and $J_2=J_3=0.18J_1$. In Fig.~\ref{material}(b), the magnetic field splits the three energy levels of a single trimer into eight levels. When $H_z \leq 1.5$, 
the ground state is $\left|0\right\rangle$ with an energy of $E_0 =-0.955 J_1 -H_z /2$. 
 Although the antiferromagnetic interaction $J_3$ competes with the interaction $J_1$ and induces a frustration in the spin system, the frustrated trimer and isolated trimer share the same eigenfunctions and quantum numbers due to the small $J_3$. Only the eigenenergies have little shifts, see Fig.~\ref{material}(c). Therefore, the spin excitations can be well described by the  quasiparticles doublons and quartons from the trimer chain without $J_3$ \cite{cheng2022}.  The Ref.~\cite{bera2022emergent}  has also confirmed the phase diagram and $1/3$ magnetic plateau under the magnetic field, but much less is  known about the  evolution of intermediate-energy and high-energy excitations   under the magnetic field. In this subsection, we present the  excitation spectrum  of the model shown in Fig.~\ref{material}(a) which is related to the material \ce{Na_2Cu_3Ge_4O_12}.  Due to the weak $J_3$,  the spin excitations  in four phases displayed in Figs.~\ref{material}(d)-(h) are also similar to the ones of trimer chain without $J_3$, such as the separation of high-energy spectra, the gapless excitations in the XY-I and XY-II phases, the gap at the edges of Brillouin zones in the Ferromagnetic phase. Our theoretical results about the high-energy quasiparticles excitations under the magnetic field can be directly  verified through the inelastic neutron scattering of the material \ce{Na_2Cu_3Ge_4O_12}. 

 Moreover, BEC is a fascinating state of matter that has been observed in bosonic atoms and cold gases. Quasiparticles in the magnetic excitations with integer spin and Bose statistics, such as the magnon and triplon, also play a crucial role in investigating   BEC \cite{ruegg2003bose,giamarchi2008bose,zapf2014bose}. Particularly in the dimerized antiferromgnets, such as \ce{TlCuCl_3}, the  intradimer interaction is the stronger than the interdimer one, thereby an isolated dimer has a singlet ground state with total spin $S=0$ and a triplet excited state with spin $S=1$. Due to the weak interdimer interaction, the magnetic excitations are dominated by triplons. When a magnetic field is applied, the Zeeman term controls the density of triplons, causing  the triplon with a magnetic quantum number $S^z=1$ to decrease in energy. At a critical magnetic field $H_{C1}$, the energy of the triplons reaches zero, and they gradually condense in the ground state until the another critical magnetic field $H_{C2}$ is reached. Beyond $H_{C2}$, all spins become polarized. Considering the characteristics of triplon BEC mentioned above,  it is natural to ask that  whether quarton BEC can be observed in trimerized systems. Here, we provide a simple analysis of the quarton BEC based on our results.  Firstly, the high-energy quartons stem from the internal trimer excitation, and possess an integer spin quantum number $S=1$, satisfying the Bose statistics. Secondly, the magnetic field brings   the lower branch of quartons closer to zero energy, as show in Fig.~\ref{Sxx-plateau-dispersion} and Figs.~\ref{material}(e)(f). At the critical point $H_{C1}$, which separates the $1/3$ magnetization plateau phase and XY-II phase, the lower quartons begin to condense and accumulate as the magnetic field increases in the XY-II phase.  Furthermore, although the BEC has been observed in the real materials featuring the 3D spin systems,  the 1D limit provides a good starting point for understanding the field-induced  quasiparticle
BEC \cite{zapf2014bose,volkov2020magnon}. Our recent study of  spin dynamics in 2D trimer systems also reveals the presence of high-energy quartons \cite{chang2023magnon}. 
Consequently,  there is a significant possibility that field-induced quarton BEC will manifest in the XY-II phase. 
 Theoretically, it is interesting to search for more evidence of quarton BEC by explore the $(T,H_z)$ phase diagram for the 1D and 2D trimer systems. Experimentally,
the material \ce{Na_2Cu_3Ge_4O_12} provides a very suitable platform for exploring the  quarton BEC.



%%%%%%%%%%%%%%%%%%%%%%%%%%%%%%%%%%%%%%%%%%%%%%%%%%%%%%%%%%%%%%%%%%%%%%%%%
\section{Discussion}\label{discussion}
In summary, we have investigated the quantum phase transitions and composite excitations of the antiferromagnetic trimer chain in a  longitudinal magnetic field by employing the ED, CPT and DMRG-TDVP methods. 
 We have demonstrated that the interplay of magnetic field and interaction leads to the  four distinct phases: XY-I, $1/3$ magnetization plateau, XY-II, and ferromagnetic phases.  By mapping the   entanglement entropy onto the parameter space $(g,H_z)$, we have obtained a detailed phase paragram, and have confirmed that the critical phases XY-I and XY-II phases are both described by the conformal field theory with the central charge $c \simeq 1$. 

 In the longitudinal excitations of trimer chain with magnetic field, we have identified  that the incommensurate wave numbers of zero energy is dependent of the magnetization in the XY-I and XY-II phases.  In addition, we have observed the presence of gapped excitations in both the $1/3$ magnetization plateau and ferromagnetic phases. Specifically, a continuum exists at the high-energy regime in the $1/3$ magnetization plateau phase. In  the ferromagnetic phases, the excitation is still described  by the spin waves,  but magnons at the Brillouin zone edges exhibit two diverse energies for the same wave vector. 


 Furthermore, we have uncovered   the intermediate-energy and high-energy excitations for small $g$, and have elucidated their excitation mechanisms    in the XY-I and $1/3$ magnetization plateau  phases by identifying  their dispersion relations. 
 In these phases,  the intermediate-energy and high-energy modes correspond to the propagating internal trimer excitations known as doublons and quartons, respectively.
  Comparing to the trimer chain without magnetic field, the high-energy spectra split into two branches, corresponding  to the upper quarton and lower quarton, respectively. As the magnetic field increases,  the gap between these two branches widens, with the lower quarton becoming the low-energy spectrum.


 Experimentally, there are already examples of coupled-trimer quantum magnets, such as   \ce{A_3Cu_3(PO_4)_4}
 (\ce{A=Ca, Sr, Pb}) \cite{PhysRevB.71.144411,drillon19931d,belik2005long,PhysRevB.76.014409} and \ce{Na_2Cu_3Ge_4O_12 } \cite{bera2022emergent}. Although  the trimers in  \ce{Pb_3Cu_3(PO_4)_4} do not exhibit a linear arrangement,
 two flat excitations at $ \omega \sim 9 meV$ and $ \omega  \sim 13.5 meV$  have been observed in the inelastic neutron-scattering spectra measured at $8K$   \cite{PhysRevB.71.144411}. These excitations  are closely related to the intermediate-energy (at $ \omega \sim J_1$) and high-energy (at $ \omega \sim 1.5 J_1$) excitations  in the trimer chain without magnetic field \cite{cheng2022}. Additionally, for the quantum magnets \ce{Na_2Cu_3Ge_4O_12},  an additional next-near neighbor interaction exists in the trimers, but its strength is weak, and  the wave functions  and quantum numbers of a single trimer remain invariant, thereby the doublons and quartons have been observed in the inelastic neutron-scattering experiments \cite{bera2022emergent}. We have theoretically demonstrated that the doublons and quartons remain observable in the trimer chain under a magnetic field even when the interaction $J_3$ is introduced.  
 These results can be directly examined in the inelastic neutron-scattering experiments measured on above quantum materials. 
 Moreover, based on the results of the 1D trimer chain under a magnetic field, it is very likely that the quarton BEC may be  identified in the experiments utilizing above  materials.
 Our results will be valuable for interpreting inelastic neutron scattering and other experiments that
 probe the high-energy excitations beyond the spin waves and spinons, as well as for facilitating detailed investigations of coexisting exotic excitations.


\section{Methods}\label{method}
\subsection{Matrix product states}

DSF is an important physical quantity for studying the spin dynamics, which  has been successfully facilitated by the DMRG along with the time evolution algorithms \cite{PhysRevLett.93.076401,PhysRevB.79.245101,PhysRevB.94.085136,PAECKEL2019167998,PhysRevLett.125.187201,PhysRevB.108.L220401}.
In this article, we primarily employ the TDVP method to  handle with  the time evolution of many-body systems \cite{PhysRevLett.107.070601, PhysRevB.94.165116}.Specifically, we perform the DMRG-TDVP calculations on a finite chain with open boundary conditions to analyze the spectrum. We denote the ground state of trimer chain with magnetic field as $\left| \mathcal{G} \right\rangle$, then we can  calculate the real time evolution of correlation function,
\begin{equation}
\left\langle \mathcal{ G} \left|S_j^{\alpha}(t) S_0^{\beta}(0)\right| \mathcal{G} \right\rangle=\mathrm{e}^{\mathrm{i} E_0 t}\left\langle \mathcal{G}\left|S_j^{\alpha} \mathrm{e}^{-\mathrm{i} \mathcal{H} t} S_0^{\beta}\right| \mathcal{G}\right\rangle,
\end{equation}
for various times $t$ and distances $j$.  $E_0$ represents the ground state energy, and we  select the site in the middle of the chain as the site  index $0$.  Firstly, we obtain the ground state $\left| \mathcal{G} \right\rangle$ in terms of a matrix product state (MPS)  employing the DMRG method. Next, we  apply the local perturbation $\hat{S}_0^\beta$ in the middle of the spin chain to generate the initial state
 \begin{equation}
	\left| \phi \right\rangle = \hat{S}_0^\beta \left|  \mathcal{G} \right\rangle
 \end{equation}
 for real-time evolution. The real-time evolution state 
  \begin{equation}
	\left| \phi(t) \right\rangle = \mathrm{e}^{-\mathrm{i} \mathcal{H} t} \left| \phi \right\rangle
  \end{equation}
 is carried out using the single-site TDVP with a time step of $dt=0.05J_1^{-1}$ and maximum time $t_{\rm{max}}=200J_1^{-1}$. Finally, we perform a Fourier transformation to obtain $\mathcal{S}^{\alpha \beta} (q, \omega)$
\begin{equation}
\mathcal{S}^{\alpha \beta}(q, \omega)=\sum_j \mathrm{e}^{-\mathrm{i} q j}\left[\int_{-\infty}^{\infty} \mathrm{d} t \mathrm{e}^{\mathrm{i} \omega t}\left\langle\hat{S}_j^\alpha(t) \hat{S}_0^\beta\right\rangle\right].
\end{equation}

Technically, in order to remove the constraint imposed by the finite-time limit on the resolution of the spectral functions in frequency space, we incorporate a Gaussian windowing function $\mathrm{exp}\left[-4(t/t_{max})^2 \right]$ in the reconstruction of the DSF \cite{PhysRevLett.93.076401}. During the DMRG calculation, we have set $\varepsilon_{\rm{SVD}}=10^{-11}$ and kept  a maximum of  $6000$ states. The time evolution is performed  on a chain with open boundary conditions and $N=120$ spins, which is  sufficiently large to avoid  finite-size effects, and the maximum bond dimension is set to $2000$. All MPS simulations are carried out using the ITensor library \cite{SciPostPhysCodeb.4}. 
 
 \subsection{ Cluster perturbation theory}
Cluster perturbation theory (CPT) is a theoretical framework used to study the electronic and magnetic properties of strongly correlated electrons~\cite{PhysRevB.48.418,PhysRevLett.84.522,RevModPhys.77.1027,PhysRevB.98.134410}, especially for calculating the single-particle spectral functions of Hubbard-type fermionic models and the dynamical spin structure factors of Heisenberg models. The basic idea behind CPT is to divide a large system into smaller clusters, calculate the properties of these clusters exactly, and then use the mean-field and perturbation theory to infer the properties of the entire system. Here, we employ ED as a solver to calculate the dynamical spin structure factor within the cluster. Following Ref.~\onlinecite{PhysRevB.98.134410}, we give a brief overview of the steps involved in cluster perturbation theory for spin models.

Firstly, we transform the spin model into a hard-core boson model using the following mapping,
\begin{eqnarray}
	S_i^+=b^\dagger, S_i^-=b, S_i^z=b_i^{\dagger}b_i-1/2,
\end{eqnarray}
Then the Hamiltonian can be rewrited as,
\begin{eqnarray}
	\mathcal{H}&=&\sum_{i=1}^{N}  \left( \frac{J_1}{2} b_{i,a}^\dagger b_{i,b} + \frac{J_1}{2} b_{i,b}^\dagger b_{i,c} + \frac{J_2}{2} b_{i,c}^\dagger b_{i+1,a} + H.c. \right) \nonumber \\
    &+&\sum_{i=1}^{N}\left[ J_1 n_{i,a} n_{i,b}+J_1 n_{i,b} n_{i,c}
	+ J_2 n_{i,c}n_{i+1,a} \right]  \nonumber \\
 &-& \left(H_z + \frac{J_1 + J_2}{2}\right) \sum_{i=1,i\in a/c}^{N} n_i \nonumber \\
	&-& \left(H_z + J_1\right) \sum_{i=1,i\in b}^{N} n_i + \mathcal{H}_\mathrm{const.},
\end{eqnarray}
where the $b_i^\dagger$, $b_i$ and $n_i=b_i^\dagger b_i$ are the bosonic operators with hard-core constraint $n_i=0$ or $1$.

Secondly, we split the system into clusters. In our calculations, the cluster size is choosen to be $N=8$, $ L=24$ which is large enough to get the accurate results. For the interaction bonds connecting nearby clusters. We use self-consistent mean-field treatment to decouple the interactions between clusters,
\begin{eqnarray}
	J_2 n_{1,c}n_{N,a}\approx J_2\left(\left\langle n_{1,a}\right\rangle n_{N,c}+\left\langle n_{N,c}\right\rangle n_{1,a}\right).
\end{eqnarray}

Thirdly, we employ exact diagonalization to self-consistently obtain the mean-field potentials of two end sites, $\left\langle n_{1,c}\right\rangle$ and $\left\langle n_{N,c}\right\rangle$. And after the convergence, we run a ED simulation to obtained the real-frequency single-particle Green function matrix $\mathbf{G}_{ij}^{C}(\omega)$ using Lanczos iteration method, where $C$ denotes the Green function matrix of cluster.

Fourthly, the original lattice Green function matrix can be obtained from the cluster Green function matrix by neglecting the nonlocal self-energy between clusters.
\begin{eqnarray}
	\mathbf{G}^{L,-1}(\tilde{q},\omega)=\mathbf{G}^{C,-1}(\omega)-V(\tilde{q}).
\end{eqnarray}
where $L$ denotes the Green function matrix of original lattice, $\tilde{\mathbf{q}}$ is the wave vector in the Brillouin zone of supercell form by the cluster.

Fifthly, we do the reperiodization of Green function matrix to restore the translational invariance. 
\begin{eqnarray}
	G_\mathrm{CPT}(q, \omega)=\frac{1}{N_s}\sum_{ij} \mathrm{e}^{-\mathrm{i} q(\mathbf{r}_i-\mathbf{r}_j)}G_{ij}^L(\omega).
\end{eqnarray}
Then the transverse dynamical spin structure factor can be obtained via
\begin{eqnarray}
	\mathcal{S}^{+-}(q, \omega)=-\frac{1}{\pi} \mathrm{Im} G_\mathrm{CPT}(q, \omega).
\end{eqnarray}

The cluster perturbation theory for spin models has been successfully applied to investigate the $J_1-J_2$ and $J_1-J_3$ models on 2D square lattice~\cite{PhysRevB.98.134410,PhysRevB.106.125129}, the $J_1-J_2$ model on honeycomb lattice~\cite{PhysRevB.105.174403}, as well as 2D trimer models~\cite{chang2023magnon}. This method proves effective in characterizing the continua in quantum spin liquid phases, as well as the magnon and triplon excitations in conventional N\'{e}el and valence bond solid phases. This method is exact in two limiting cases. One is the interactions between clusters tend towards zero or are extremely weak; Another is the cluster size approaches infinity or is very large. In the case of our trimer chain model, when $g=J_2/J_1$ is small, accurate results can be obtained even with very small clusters, such as $N=2,4$. The choice of a cluster size of $N=8$ in our study ensures accuracy in both small and large $g$ regimes. This cluster size proves sufficiently large to guarantee precision across a wide range of parameter values.

\begin{acknowledgments}
This project is supported by the National Key R$\&$D Program of China, Grants No. 2022YFA1402802, No. 2018YFA0306001, NSFC-92165204, NSFC-11974432, and Shenzhen Institute for Quantum Science and Engineering (Grant No. SIQSE202102). J.Q.C. is supported by the National Natural Science Foundation of China through Grants No. 12047562.  H.Q.W. is supported by the National Natural Science Foundation of China through Grants No. 11804401, No. 11832019 and the Guangzhou Basic and Applied Basic Research Foundation (202201011569). 
\end{acknowledgments}

\section{ AUTHOR CONTRIBUTIONS}
 D.X.Y., H.Q.W., and J.Q.C. conceived and designed the project. J.Q.C., H.Q.W. and Z.Y.N. performed the numerical simulations.  J.Q.C., H.Q.W., Z.Y.N., and D.X.Y. provided the explanation of the numerical results. All authors contributed to the discussion of the results and wrote the paper.

\section{ Data Availability}
The data that support the findings of this study are available from the corresponding authors upon reasonable request.


\section{Code availability}
The code used for the analysis is available from the authors upon reasonable request.


\section{ Competing Interests}
The authors declare no competing interests.








%\nocite{*}

%\bibliography{apssamp}% Produces the bibliography via BibTeX.
\bibliographystyle{naturemag}

\begin{thebibliography}{10}
  \expandafter\ifx\csname url\endcsname\relax
    \def\url#1{\texttt{#1}}\fi
  \expandafter\ifx\csname urlprefix\endcsname\relax\def\urlprefix{URL }\fi
  \providecommand{\bibinfo}[2]{#2}
  \providecommand{\eprint}[2][]{\url{#2}}
  
  \bibitem{Mikeska2004}
  \bibinfo{author}{Mikeska, H.-J.} \& \bibinfo{author}{Kolezhuk, A.~K.}
  \newblock \emph{\bibinfo{title}{One-dimensional magnetism}},
    \bibinfo{pages}{1--83} (\bibinfo{publisher}{Springer Berlin Heidelberg},
    \bibinfo{address}{Berlin, Heidelberg}, \bibinfo{year}{2004}).
  
  \bibitem{karbach97}
  \bibinfo{author}{Karbach, M.}, \bibinfo{author}{M\"uller, G.},
    \bibinfo{author}{Bougourzi, A.~H.}, \bibinfo{author}{Fledderjohann, A.} \&
    \bibinfo{author}{M\"utter, K.-H.}
  \newblock \bibinfo{title}{Two-spinon dynamic structure factor of the
    one-dimensional $\mathrm{S}=1/2$ {Heisenberg} antiferromagnet}.
  \newblock \emph{\bibinfo{journal}{Phys. Rev. B}} \textbf{\bibinfo{volume}{55}},
    \bibinfo{pages}{12510--12517} (\bibinfo{year}{1997}).
  
  \bibitem{tennant93}
  \bibinfo{author}{Tennant, D.~A.}, \bibinfo{author}{Perring, T.~G.},
    \bibinfo{author}{Cowley, R.~A.} \& \bibinfo{author}{Nagler, S.~E.}
  \newblock \bibinfo{title}{Unbound spinons in the $\mathrm{S}=1/2$
    antiferromagnetic chain $\mathrm{KCuF}_3$}.
  \newblock \emph{\bibinfo{journal}{Phys. Rev. Lett}}
    \textbf{\bibinfo{volume}{70}}, \bibinfo{pages}{4003} (\bibinfo{year}{1993}).
  
  \bibitem{lake13}
  \bibinfo{author}{Lake, B.} \emph{et~al.}
  \newblock \bibinfo{title}{Multispinon continua at zero and finite temperature
    in a near-ideal {Heisenberg} chain}.
  \newblock \emph{\bibinfo{journal}{Phys. Rev. Lett.}}
    \textbf{\bibinfo{volume}{111}}, \bibinfo{pages}{137205}
    (\bibinfo{year}{2013}).
  
  \bibitem{PhysRevLett.104.237207}
  \bibinfo{author}{Enderle, M.} \emph{et~al.}
  \newblock \bibinfo{title}{Two-spinon and four-spinon continuum in a frustrated
    ferromagnetic spin-$1/2$ chain}.
  \newblock \emph{\bibinfo{journal}{Phys. Rev. Lett.}}
    \textbf{\bibinfo{volume}{104}}, \bibinfo{pages}{237207}
    (\bibinfo{year}{2010}).
  
  \bibitem{PhysRevLett.106.157205}
  \bibinfo{author}{Klauser, A.}, \bibinfo{author}{Mossel, J.},
    \bibinfo{author}{Caux, J.-S.} \& \bibinfo{author}{van~den Brink, J.}
  \newblock \bibinfo{title}{Spin-exchange dynamical structure factor of the
    {$S=1/2$} {Heisenberg} chain}.
  \newblock \emph{\bibinfo{journal}{Phys. Rev. Lett.}}
    \textbf{\bibinfo{volume}{106}}, \bibinfo{pages}{157205}
    (\bibinfo{year}{2011}).
  
  \bibitem{RIXS2018NC}
  \bibinfo{author}{Schlappa, J.} \emph{et~al.}
  \newblock \bibinfo{title}{Probing multi-spinon excitations outside of the
    two-spinon continuum in the antiferromagnetic spin chain cuprate
    {$\mathrm{Sr}_2 \mathrm{CuO}_3$}}.
  \newblock \emph{\bibinfo{journal}{Nat. Commun.}} \textbf{\bibinfo{volume}{9}},
    \bibinfo{pages}{5394} (\bibinfo{year}{2018}).
  
  \bibitem{PhysRevLett.102.037203}
  \bibinfo{author}{Kohno, M.}
  \newblock \bibinfo{title}{Dynamically dominant excitations of string solutions
    in the spin-$1/2$ antiferromagnetic {Heisenberg} chain in a magnetic field}.
  \newblock \emph{\bibinfo{journal}{Phys. Rev. Lett.}}
    \textbf{\bibinfo{volume}{102}}, \bibinfo{pages}{037203}
    (\bibinfo{year}{2009}).
  
  \bibitem{wang2018experimental}
  \bibinfo{author}{Wang, Z.} \emph{et~al.}
  \newblock \bibinfo{title}{Experimental observation of {Bethe} strings}.
  \newblock \emph{\bibinfo{journal}{Nature}} \textbf{\bibinfo{volume}{554}},
    \bibinfo{pages}{219--223} (\bibinfo{year}{2018}).
  
  \bibitem{PRL2019ZheWang}
  \bibinfo{author}{Wang, Z.} \emph{et~al.}
  \newblock \bibinfo{title}{Quantum critical dynamics of a {Heisenberg-Ising}
    chain in a longitudinal field: Many-body strings versus fractional
    excitations}.
  \newblock \emph{\bibinfo{journal}{Phys. Rev. Lett.}}
    \textbf{\bibinfo{volume}{123}}, \bibinfo{pages}{067202}
    (\bibinfo{year}{2019}).
  
  \bibitem{dagotto96}
  \bibinfo{author}{Dagotto, E.} \& \bibinfo{author}{Rice, T.~M.}
  \newblock \bibinfo{title}{Surprises on the way from one- to two-dimensional
    quantum magnets: The ladder materials}.
  \newblock \emph{\bibinfo{journal}{Science}} \textbf{\bibinfo{volume}{271}},
    \bibinfo{pages}{618} (\bibinfo{year}{1996}).
  
  \bibitem{haldane1983}
  \bibinfo{author}{Haldane, F. D.~M.}
  \newblock \bibinfo{title}{Continuum dynamics of the {1-D} {Heisenberg}
    antiferromagnet: Identification with the {O}(3) nonlinear sigma model}.
  \newblock \emph{\bibinfo{journal}{Phys. Lett. A}}
    \textbf{\bibinfo{volume}{93}}, \bibinfo{pages}{464--468}
    (\bibinfo{year}{1983}).
  
  \bibitem{schmidiger13}
  \bibinfo{author}{Schmidiger, D.} \emph{et~al.}
  \newblock \bibinfo{title}{Symmetric and asymmetric excitations of a strong-leg
    quantum spin ladder}.
  \newblock \emph{\bibinfo{journal}{Phys. Rev. B}} \textbf{\bibinfo{volume}{88}},
    \bibinfo{pages}{094411} (\bibinfo{year}{2013}).
  
  \bibitem{doretto09}
  \bibinfo{author}{Doretto, R.~L.} \& \bibinfo{author}{Vojta, M.}
  \newblock \bibinfo{title}{Quantum magnets with weakly confined spinons:
    Multiple length scales and quantum impurities}.
  \newblock \emph{\bibinfo{journal}{Phys. Rev. B}} \textbf{\bibinfo{volume}{80}},
    \bibinfo{pages}{024411} (\bibinfo{year}{2009}).
  
  \bibitem{cheng2022}
  \bibinfo{author}{Cheng, J.-Q.} \emph{et~al.}
  \newblock \bibinfo{title}{Fractional and composite excitations of
    antiferromagnetic quantum spin trimer chains}.
  \newblock \emph{\bibinfo{journal}{npj Quantum Mater.}}
    \textbf{\bibinfo{volume}{7}}, \bibinfo{pages}{1--11} (\bibinfo{year}{2022}).
  
  \bibitem{PhysRevB.71.144411}
  \bibinfo{author}{Matsuda, M.} \emph{et~al.}
  \newblock \bibinfo{title}{Magnetic excitations from the linear {Heisenberg}
    antiferromagnetic spin trimer system
    {${A}_{3}\mathrm{Cu}_{3}{(\mathrm{P}{\mathrm{O}}_{4})}_{4}$
    ($\mathrm{A}=\mathrm{Ca}$,$\mathrm{Sr}$, and $\mathrm{Pb}$)}}.
  \newblock \emph{\bibinfo{journal}{Phys. Rev. B}} \textbf{\bibinfo{volume}{71}},
    \bibinfo{pages}{144411} (\bibinfo{year}{2005}).
  
  \bibitem{drillon19931d}
  \bibinfo{author}{Drillon, M.} \emph{et~al.}
  \newblock \bibinfo{title}{{1D} ferrimagnetism in copper(ii) trimetric chains:
    Specific heat and magnetic behavior of {$A_3 Cu_3 (PO_4)_4$} with {$A = Ca,
    Sr$}}.
  \newblock \emph{\bibinfo{journal}{J. Magn. Magn. Mater.}}
    \textbf{\bibinfo{volume}{128}}, \bibinfo{pages}{83--92}
    (\bibinfo{year}{1993}).
  
  \bibitem{belik2005long}
  \bibinfo{author}{Belik, A.~A.}, \bibinfo{author}{Matsuo, A.},
    \bibinfo{author}{Azuma, M.}, \bibinfo{author}{Kindo, K.} \&
    \bibinfo{author}{Takano, M.}
  \newblock \bibinfo{title}{Long-range magnetic ordering of s= 1/2 linear trimers
    in {${A}_{3}\mathrm{Cu}_{3}{(\mathrm{P}{\mathrm{O}}_{4})}_{4}$
    ($\mathrm{A}=\mathrm{Ca}$,$\mathrm{Sr}$, $\mathrm{Pb}$)}}.
  \newblock \emph{\bibinfo{journal}{J. Solid State Chem.}}
    \textbf{\bibinfo{volume}{178}}, \bibinfo{pages}{709--714}
    (\bibinfo{year}{2005}).
  
  \bibitem{PhysRevB.76.014409}
  \bibinfo{author}{Yamamoto, S.} \& \bibinfo{author}{Ohara, J.}
  \newblock \bibinfo{title}{Low-energy structure of the homometallic intertwining
    double-chain ferrimagnets
    {${A}_{3}\mathrm{Cu}_{3}{(\mathrm{P}{\mathrm{O}}_{4})}_{4}$
    $(\mathrm{A}=\mathrm{Ca},\mathrm{Sr},\mathrm{Pb})$}}.
  \newblock \emph{\bibinfo{journal}{Phys. Rev. B}} \textbf{\bibinfo{volume}{76}},
    \bibinfo{pages}{014409} (\bibinfo{year}{2007}).
  
  \bibitem{PhysRevB.102.035137}
  \bibinfo{author}{Montenegro-Filho, R.~R.}, \bibinfo{author}{Matias, F.~S.} \&
    \bibinfo{author}{Coutinho-Filho, M.~D.}
  \newblock \bibinfo{title}{Topology of many-body edge and extended quantum
    states in an open spin chain: 1/3 plateau, {Kosterlitz-Thouless} transition,
    and {Luttinger} liquid}.
  \newblock \emph{\bibinfo{journal}{Phys. Rev. B}}
    \textbf{\bibinfo{volume}{102}}, \bibinfo{pages}{035137}
    (\bibinfo{year}{2020}).
  
  \bibitem{PhysRevB.105.134423}
  \bibinfo{author}{Montenegro-Filho, R.~R.}, \bibinfo{author}{Silva-J\'unior, E.
    J.~P.} \& \bibinfo{author}{Coutinho-Filho, M.~D.}
  \newblock \bibinfo{title}{Ground-state phase diagram and thermodynamics of
    coupled trimer chains}.
  \newblock \emph{\bibinfo{journal}{Phys. Rev. B}}
    \textbf{\bibinfo{volume}{105}}, \bibinfo{pages}{134423}
    (\bibinfo{year}{2022}).
  
  \bibitem{hasegawa2012magnetic}
  \bibinfo{author}{Hasegawa, Y.} \& \bibinfo{author}{Matsumoto, M.}
  \newblock \bibinfo{title}{Magnetic excitation in interacting spin trimer
    systems investigated by extended spin-wave theory}.
  \newblock \emph{\bibinfo{journal}{J. Phys. Soc. Jpn.}}
    \textbf{\bibinfo{volume}{81}}, \bibinfo{pages}{094712}
    (\bibinfo{year}{2012}).
  
  \bibitem{PhysRevLett.129.207201}
  \bibinfo{author}{Shen, Y.} \emph{et~al.}
  \newblock \bibinfo{title}{Emergence of spinons in layered trimer {Iridate}
    {${\mathrm{Ba}}_{4}{\mathrm{Ir}}_{3}{\mathrm{O}}_{10}$}}.
  \newblock \emph{\bibinfo{journal}{Phys. Rev. Lett.}}
    \textbf{\bibinfo{volume}{129}}, \bibinfo{pages}{207201}
    (\bibinfo{year}{2022}).
  
  \bibitem{bera2022emergent}
  \bibinfo{author}{Bera, A.~K.} \emph{et~al.}
  \newblock \bibinfo{title}{Emergent many-body composite excitations of
    interacting spin-1/2 trimers}.
  \newblock \emph{\bibinfo{journal}{Nat. Commun.}} \textbf{\bibinfo{volume}{13}},
    \bibinfo{pages}{6888} (\bibinfo{year}{2022}).
  
  \bibitem{PhysRevLett.105.247001}
  \bibinfo{author}{Headings, N.~S.}, \bibinfo{author}{Hayden, S.~M.},
    \bibinfo{author}{Coldea, R.} \& \bibinfo{author}{Perring, T.~G.}
  \newblock \bibinfo{title}{Anomalous high-energy spin excitations in the
    {High-${T}_{c}$ } superconductor-parent antiferromagnet
    {$\mathrm{La}_{2}\mathrm{CuO}_{4}$}}.
  \newblock \emph{\bibinfo{journal}{Phys. Rev. Lett.}}
    \textbf{\bibinfo{volume}{105}}, \bibinfo{pages}{247001}
    (\bibinfo{year}{2010}).
  
  \bibitem{Zhou2013}
  \bibinfo{author}{Zhou, K.-J.} \emph{et~al.}
  \newblock \bibinfo{title}{Persistent high-energy spin excitations in
    {iron}-pnictide superconductors}.
  \newblock \emph{\bibinfo{journal}{Nat. Commun.}} \textbf{\bibinfo{volume}{4}},
    \bibinfo{pages}{1470} (\bibinfo{year}{2013}).
  
  \bibitem{Ishii2014}
  \bibinfo{author}{Ishii, K.} \emph{et~al.}
  \newblock \bibinfo{title}{High-energy spin and charge excitations in
    electron-doped copper {oxide} superconductors}.
  \newblock \emph{\bibinfo{journal}{Nat. Commun.}} \textbf{\bibinfo{volume}{5}},
    \bibinfo{pages}{3714} (\bibinfo{year}{2014}).
  
  \bibitem{PhysRevB.91.184513}
  \bibinfo{author}{Wakimoto, S.} \emph{et~al.}
  \newblock \bibinfo{title}{High-energy magnetic excitations in overdoped
    {$\mathrm{La}_{2-x}\mathrm{Sr}_{x}\mathrm{CuO}_{4}$} studied by neutron and
    resonant inelastic {X-ray} scattering}.
  \newblock \emph{\bibinfo{journal}{Phys. Rev. B}} \textbf{\bibinfo{volume}{91}},
    \bibinfo{pages}{184513} (\bibinfo{year}{2015}).
  
  \bibitem{Song2021}
  \bibinfo{author}{Song, Y.} \emph{et~al.}
  \newblock \bibinfo{title}{High-energy magnetic excitations from heavy
    quasiparticles in {$\mathrm{CeCu}_{2}\mathrm{Si}_{2}$}}.
  \newblock \emph{\bibinfo{journal}{npj Quantum Mater.}}
    \textbf{\bibinfo{volume}{6}}, \bibinfo{pages}{60} (\bibinfo{year}{2021}).
  
  \bibitem{SAC2}
  \bibinfo{author}{Shao, H.} \emph{et~al.}
  \newblock \bibinfo{title}{Nearly deconfined spinon excitations in the
    square-lattice spin-$1/2$ {Heisenberg} antiferromagnet}.
  \newblock \emph{\bibinfo{journal}{Phys. Rev. X}} \textbf{\bibinfo{volume}{7}},
    \bibinfo{pages}{041072} (\bibinfo{year}{2017}).
  
  \bibitem{dalla2015fractional}
  \bibinfo{author}{Dalla~Piazza, B.} \emph{et~al.}
  \newblock \bibinfo{title}{Fractional excitations in the square-lattice quantum
    antiferromagnet}.
  \newblock \emph{\bibinfo{journal}{Nat. Phys.}} \textbf{\bibinfo{volume}{11}},
    \bibinfo{pages}{62--68} (\bibinfo{year}{2015}).
  
  \bibitem{PhysRevB.108.224418}
  \bibinfo{author}{Gu, C.}, \bibinfo{author}{Gu, Z.-L.}, \bibinfo{author}{Yu,
    S.-L.} \& \bibinfo{author}{Li, J.-X.}
  \newblock \bibinfo{title}{Spectral evolution of the $s=\frac{1}{2}$
    antiferromagnetic {Heisenberg} model: From one to two dimensions}.
  \newblock \emph{\bibinfo{journal}{Phys. Rev. B}}
    \textbf{\bibinfo{volume}{108}}, \bibinfo{pages}{224418}
    (\bibinfo{year}{2023}).
  
  \bibitem{chang2023magnon}
  \bibinfo{author}{Chang, Y.-Y.}, \bibinfo{author}{Cheng, J.-Q.},
    \bibinfo{author}{Shao, H.}, \bibinfo{author}{Yao, D.-X.} \&
    \bibinfo{author}{Wu, H.-Q.}
  \newblock \bibinfo{title}{Magnon, doublon and quarton excitations in {2D}
    trimerized {Heisenberg} models}.
  \newblock \emph{\bibinfo{journal}{arXiv}} \textbf{\bibinfo{volume}{2401.00376}}
    (\bibinfo{year}{2023}).
  
  \bibitem{PhysRevLett.69.2863}
  \bibinfo{author}{White, S.~R.}
  \newblock \bibinfo{title}{Density matrix formulation for quantum
    renormalization groups}.
  \newblock \emph{\bibinfo{journal}{Phys. Rev. Lett.}}
    \textbf{\bibinfo{volume}{69}}, \bibinfo{pages}{2863--2866}
    (\bibinfo{year}{1992}).
  
  \bibitem{PhysRevLett.93.076401}
  \bibinfo{author}{White, S.~R.} \& \bibinfo{author}{Feiguin, A.~E.}
  \newblock \bibinfo{title}{Real-time evolution using the density matrix
    renormalization group}.
  \newblock \emph{\bibinfo{journal}{Phys. Rev. Lett.}}
    \textbf{\bibinfo{volume}{93}}, \bibinfo{pages}{076401}
    (\bibinfo{year}{2004}).
  
  \bibitem{SCHOLLWOCK201196}
  \bibinfo{author}{Schollwöck, U.}
  \newblock \bibinfo{title}{The density-matrix renormalization group in the age
    of matrix product states}.
  \newblock \emph{\bibinfo{journal}{Ann. Phys.}} \textbf{\bibinfo{volume}{326}},
    \bibinfo{pages}{96 -- 192} (\bibinfo{year}{2011}).
  
  \bibitem{PhysRevLett.107.070601}
  \bibinfo{author}{Haegeman, J.} \emph{et~al.}
  \newblock \bibinfo{title}{Time-dependent variational principle for quantum
    lattices}.
  \newblock \emph{\bibinfo{journal}{Phys. Rev. Lett.}}
    \textbf{\bibinfo{volume}{107}}, \bibinfo{pages}{070601}
    (\bibinfo{year}{2011}).
  
  \bibitem{PhysRevB.94.165116}
  \bibinfo{author}{Haegeman, J.}, \bibinfo{author}{Lubich, C.},
    \bibinfo{author}{Oseledets, I.}, \bibinfo{author}{Vandereycken, B.} \&
    \bibinfo{author}{Verstraete, F.}
  \newblock \bibinfo{title}{Unifying time evolution and optimization with matrix
    product states}.
  \newblock \emph{\bibinfo{journal}{Phys. Rev. B}} \textbf{\bibinfo{volume}{94}},
    \bibinfo{pages}{165116} (\bibinfo{year}{2016}).
  
  \bibitem{PhysRevB.48.418}
  \bibinfo{author}{Gros, C.} \& \bibinfo{author}{Valent\'{\i}, R.}
  \newblock \bibinfo{title}{Cluster expansion for the self-energy: A simple
    many-body method for interpreting the photoemission spectra of correlated
    fermi systems}.
  \newblock \emph{\bibinfo{journal}{Phys. Rev. B}} \textbf{\bibinfo{volume}{48}},
    \bibinfo{pages}{418--425} (\bibinfo{year}{1993}).
  
  \bibitem{PhysRevLett.84.522}
  \bibinfo{author}{S\'en\'echal, D.}, \bibinfo{author}{Perez, D.} \&
    \bibinfo{author}{Pioro-Ladri\`ere, M.}
  \newblock \bibinfo{title}{Spectral weight of the hubbard model through cluster
    perturbation theory}.
  \newblock \emph{\bibinfo{journal}{Phys. Rev. Lett.}}
    \textbf{\bibinfo{volume}{84}}, \bibinfo{pages}{522--525}
    (\bibinfo{year}{2000}).
  
  \bibitem{RevModPhys.77.1027}
  \bibinfo{author}{Maier, T.}, \bibinfo{author}{Jarrell, M.},
    \bibinfo{author}{Pruschke, T.} \& \bibinfo{author}{Hettler, M.~H.}
  \newblock \bibinfo{title}{Quantum cluster theories}.
  \newblock \emph{\bibinfo{journal}{Rev. Mod. Phys.}}
    \textbf{\bibinfo{volume}{77}}, \bibinfo{pages}{1027--1080}
    (\bibinfo{year}{2005}).
  
  \bibitem{PhysRevB.98.134410}
  \bibinfo{author}{Yu, S.-L.}, \bibinfo{author}{Wang, W.}, \bibinfo{author}{Dong,
    Z.-Y.}, \bibinfo{author}{Yao, Z.-J.} \& \bibinfo{author}{Li, J.-X.}
  \newblock \bibinfo{title}{Deconfinement of spinons in frustrated spin systems:
    Spectral perspective}.
  \newblock \emph{\bibinfo{journal}{Phys. Rev. B}} \textbf{\bibinfo{volume}{98}},
    \bibinfo{pages}{134410} (\bibinfo{year}{2018}).
  
  \bibitem{RevModPhys.80.517}
  \bibinfo{author}{Amico, L.}, \bibinfo{author}{Fazio, R.},
    \bibinfo{author}{Osterloh, A.} \& \bibinfo{author}{Vedral, V.}
  \newblock \bibinfo{title}{Entanglement in many-body systems}.
  \newblock \emph{\bibinfo{journal}{Rev. Mod. Phys.}}
    \textbf{\bibinfo{volume}{80}}, \bibinfo{pages}{517--576}
    (\bibinfo{year}{2008}).
  
  \bibitem{LAFLORENCIE20161}
  \bibinfo{author}{Laflorencie, N.}
  \newblock \bibinfo{title}{Quantum entanglement in condensed matter systems}.
  \newblock \emph{\bibinfo{journal}{Phys. Rep.}} \textbf{\bibinfo{volume}{646}},
    \bibinfo{pages}{1--59} (\bibinfo{year}{2016}).
  \newblock \bibinfo{note}{Quantum entanglement in condensed matter systems}.
  
  \bibitem{cheng2017}
  \bibinfo{author}{Cheng, J.-Q.}, \bibinfo{author}{Wu, W.} \&
    \bibinfo{author}{Xu, J.-B.}
  \newblock \bibinfo{title}{Multipartite entanglement in an {XXZ} spin chain with
    {Dzyaloshinskii--Moriya} interaction and quantum phase transition}.
  \newblock \emph{\bibinfo{journal}{Quantum Inf. Process.}}
    \textbf{\bibinfo{volume}{16}}, \bibinfo{pages}{1--20} (\bibinfo{year}{2017}).
  
  \bibitem{PhysRevLett.120.200602}
  \bibinfo{author}{Goldstein, M.} \& \bibinfo{author}{Sela, E.}
  \newblock \bibinfo{title}{Symmetry-resolved entanglement in many-body systems}.
  \newblock \emph{\bibinfo{journal}{Phys. Rev. Lett.}}
    \textbf{\bibinfo{volume}{120}}, \bibinfo{pages}{200602}
    (\bibinfo{year}{2018}).
  
  \bibitem{PhysRevE.97.062134}
  \bibinfo{author}{Cheng, J.-Q.} \& \bibinfo{author}{Xu, J.-B.}
  \newblock \bibinfo{title}{Multipartite entanglement, quantum coherence, and
    quantum criticality in triangular and {Sierpi\ifmmode \acute{n}\else
    \'{n}\fi{}ski } fractal lattices}.
  \newblock \emph{\bibinfo{journal}{Phys. Rev. E}} \textbf{\bibinfo{volume}{97}},
    \bibinfo{pages}{062134} (\bibinfo{year}{2018}).
  
  \bibitem{PhysRevLett.128.020402}
  \bibinfo{author}{Kunkel, P.} \emph{et~al.}
  \newblock \bibinfo{title}{Detecting entanglement structure in continuous
    many-body quantum systems}.
  \newblock \emph{\bibinfo{journal}{Phys. Rev. Lett.}}
    \textbf{\bibinfo{volume}{128}}, \bibinfo{pages}{020402}
    (\bibinfo{year}{2022}).
  
  \bibitem{PhysRevLett.78.1984}
  \bibinfo{author}{Oshikawa, M.}, \bibinfo{author}{Yamanaka, M.} \&
    \bibinfo{author}{Affleck, I.}
  \newblock \bibinfo{title}{Magnetization plateaus in spin chains: ``{Haldane}
    gap'' for half-integer spins}.
  \newblock \emph{\bibinfo{journal}{Phys. Rev. Lett.}}
    \textbf{\bibinfo{volume}{78}}, \bibinfo{pages}{1984--1987}
    (\bibinfo{year}{1997}).
  
  \bibitem{PhysRevB.65.214403}
  \bibinfo{author}{Sakai, T.} \& \bibinfo{author}{Okamoto, K.}
  \newblock \bibinfo{title}{Quantum magnetization plateaux of an anisotropic
    ferrimagnetic spin chain}.
  \newblock \emph{\bibinfo{journal}{Phys. Rev. B}} \textbf{\bibinfo{volume}{65}},
    \bibinfo{pages}{214403} (\bibinfo{year}{2002}).
  
  \bibitem{PhysRevLett.104.095701}
  \bibinfo{author}{Calabrese, P.}, \bibinfo{author}{Campostrini, M.},
    \bibinfo{author}{Essler, F.} \& \bibinfo{author}{Nienhuis, B.}
  \newblock \bibinfo{title}{Parity effects in the scaling of block entanglement
    in gapless spin chains}.
  \newblock \emph{\bibinfo{journal}{Phys. Rev. Lett.}}
    \textbf{\bibinfo{volume}{104}}, \bibinfo{pages}{095701}
    (\bibinfo{year}{2010}).
  
  \bibitem{PhysRevB.92.054411}
  \bibinfo{author}{D'Emidio, J.}, \bibinfo{author}{Block, M.~S.} \&
    \bibinfo{author}{Kaul, R.~K.}
  \newblock \bibinfo{title}{R\'enyi entanglement entropy of critical
    {$\mathrm{SU}(N)$} spin chains}.
  \newblock \emph{\bibinfo{journal}{Phys. Rev. B}} \textbf{\bibinfo{volume}{92}},
    \bibinfo{pages}{054411} (\bibinfo{year}{2015}).
  
  \bibitem{PhysRevB.105.014435}
  \bibinfo{author}{Feng, S.}, \bibinfo{author}{Alvarez, G.} \&
    \bibinfo{author}{Trivedi, N.}
  \newblock \bibinfo{title}{Gapless to gapless phase transitions in quantum spin
    chains}.
  \newblock \emph{\bibinfo{journal}{Phys. Rev. B}}
    \textbf{\bibinfo{volume}{105}}, \bibinfo{pages}{014435}
    (\bibinfo{year}{2022}).
  
  \bibitem{takayoshi2023phase}
  \bibinfo{author}{Takayoshi, S.} \emph{et~al.}
  \newblock \bibinfo{title}{Phase transitions and spin dynamics of the quasi-one
    dimensional {Ising}-like antiferromagnet {BaCo$_{2}$V$_{2}$O$_{8}$} in a
    longitudinal magnetic field} (\bibinfo{year}{2023}).
  \newblock \eprint{2302.03833}.
  
  \bibitem{PhysRevB.94.125130}
  \bibinfo{author}{Wang, Z.} \emph{et~al.}
  \newblock \bibinfo{title}{From confined spinons to emergent fermions:
    Observation of elementary magnetic excitations in a transverse-field ising
    chain}.
  \newblock \emph{\bibinfo{journal}{Phys. Rev. B}} \textbf{\bibinfo{volume}{94}},
    \bibinfo{pages}{125130} (\bibinfo{year}{2016}).
  
  \bibitem{JAP2014}
  \bibinfo{author}{Yasui, Y.}, \bibinfo{author}{Kawamura, Y.},
    \bibinfo{author}{Kobayashi, Y.} \& \bibinfo{author}{Sato, M.}
  \newblock \bibinfo{title}{{Magnetic and dielectric properties of
    one-dimensional array of $S=1/2$ linear trimer system {$\mathrm{Na}_2
    \mathrm{Cu}_3 \mathrm{Ge}_4 \mathrm{O}_{12}$}}}.
  \newblock \emph{\bibinfo{journal}{J. Appl. Phys.}}
    \textbf{\bibinfo{volume}{115}}, \bibinfo{pages}{17E125}
    (\bibinfo{year}{2014}).
  
  \bibitem{ruegg2003bose}
  \bibinfo{author}{R{\"u}egg, C.} \emph{et~al.}
  \newblock \bibinfo{title}{{Bose--Einste} in condensation of the triplet states
    in the magnetic insulator {$\mathrm{TlCuCl}_3$}}.
  \newblock \emph{\bibinfo{journal}{Nature}} \textbf{\bibinfo{volume}{423}},
    \bibinfo{pages}{62--65} (\bibinfo{year}{2003}).
  
  \bibitem{giamarchi2008bose}
  \bibinfo{author}{Giamarchi, T.}, \bibinfo{author}{R{\"u}egg, C.} \&
    \bibinfo{author}{Tchernyshyov, O.}
  \newblock \bibinfo{title}{{Bose--Einstein} condensation in magnetic
    insulators}.
  \newblock \emph{\bibinfo{journal}{Nat. Phys.}} \textbf{\bibinfo{volume}{4}},
    \bibinfo{pages}{198--204} (\bibinfo{year}{2008}).
  
  \bibitem{zapf2014bose}
  \bibinfo{author}{Zapf, V.}, \bibinfo{author}{Jaime, M.} \&
    \bibinfo{author}{Batista, C.}
  \newblock \bibinfo{title}{{Bose-Einstein} condensation in quantum magnets}.
  \newblock \emph{\bibinfo{journal}{Rev. Mod. Phys.}}
    \textbf{\bibinfo{volume}{86}}, \bibinfo{pages}{563} (\bibinfo{year}{2014}).
  
  \bibitem{volkov2020magnon}
  \bibinfo{author}{Volkov, P.~A.}, \bibinfo{author}{Gazit, S.} \&
    \bibinfo{author}{Pixley, J.~H.}
  \newblock \bibinfo{title}{Magnon {Bose--Einstein} condensation and
    superconductivity in a frustrated kondo lattice}.
  \newblock \emph{\bibinfo{journal}{Proc. Nat. Acad. Sci.}}
    \textbf{\bibinfo{volume}{117}}, \bibinfo{pages}{20462--20467}
    (\bibinfo{year}{2020}).
  
  \bibitem{PhysRevB.79.245101}
  \bibinfo{author}{Barthel, T.}, \bibinfo{author}{Schollw\"ock, U.} \&
    \bibinfo{author}{White, S.~R.}
  \newblock \bibinfo{title}{Spectral functions in one-dimensional quantum systems
    at finite temperature using the density matrix renormalization group}.
  \newblock \emph{\bibinfo{journal}{Phys. Rev. B}} \textbf{\bibinfo{volume}{79}},
    \bibinfo{pages}{245101} (\bibinfo{year}{2009}).
  
  \bibitem{PhysRevB.94.085136}
  \bibinfo{author}{Bruognolo, B.}, \bibinfo{author}{Weichselbaum, A.},
    \bibinfo{author}{von Delft, J.} \& \bibinfo{author}{Garst, M.}
  \newblock \bibinfo{title}{Dynamic structure factor of the spin-$\frac{1}{2}$
    {XXZ} chain in a transverse field}.
  \newblock \emph{\bibinfo{journal}{Phys. Rev. B}} \textbf{\bibinfo{volume}{94}},
    \bibinfo{pages}{085136} (\bibinfo{year}{2016}).
  
  \bibitem{PAECKEL2019167998}
  \bibinfo{author}{Paeckel, S.} \emph{et~al.}
  \newblock \bibinfo{title}{Time-evolution methods for matrix-product states}.
  \newblock \emph{\bibinfo{journal}{Ann. Phys.}} \textbf{\bibinfo{volume}{411}},
    \bibinfo{pages}{167998} (\bibinfo{year}{2019}).
  
  \bibitem{PhysRevLett.125.187201}
  \bibinfo{author}{Keselman, A.}, \bibinfo{author}{Balents, L.} \&
    \bibinfo{author}{Starykh, O.~A.}
  \newblock \bibinfo{title}{Dynamical signatures of quasiparticle interactions in
    quantum spin chains}.
  \newblock \emph{\bibinfo{journal}{Phys. Rev. Lett.}}
    \textbf{\bibinfo{volume}{125}}, \bibinfo{pages}{187201}
    (\bibinfo{year}{2020}).
  
  \bibitem{PhysRevB.108.L220401}
  \bibinfo{author}{Drescher, M.}, \bibinfo{author}{Vanderstraeten, L.},
    \bibinfo{author}{Moessner, R.} \& \bibinfo{author}{Pollmann, F.}
  \newblock \bibinfo{title}{Dynamical signatures of symmetry-broken and liquid
    phases in an $s$ = $\frac{1}{2}$ {Heisenberg} antiferromagnet on the
    triangular lattice}.
  \newblock \emph{\bibinfo{journal}{Phys. Rev. B}}
    \textbf{\bibinfo{volume}{108}}, \bibinfo{pages}{L220401}
    (\bibinfo{year}{2023}).
  
  \bibitem{SciPostPhysCodeb.4}
  \bibinfo{author}{Fishman, M.}, \bibinfo{author}{White, S.~R.} \&
    \bibinfo{author}{Stoudenmire, E.~M.}
  \newblock \bibinfo{title}{{The {ITensor} Software Library for Tensor Network
    Calculations}}.
  \newblock \emph{\bibinfo{journal}{SciPost Phys. Codebases}} \bibinfo{pages}{4}
    (\bibinfo{year}{2022}).
  
  \bibitem{PhysRevB.106.125129}
  \bibinfo{author}{Wu, M.}, \bibinfo{author}{Gong, S.-S.}, \bibinfo{author}{Yao,
    D.-X.} \& \bibinfo{author}{Wu, H.-Q.}
  \newblock \bibinfo{title}{Phase diagram and magnetic excitations of
    ${J}_{1}-{J}_{3}$ {Heisenberg} model on the square lattice}.
  \newblock \emph{\bibinfo{journal}{Phys. Rev. B}}
    \textbf{\bibinfo{volume}{106}}, \bibinfo{pages}{125129}
    (\bibinfo{year}{2022}).
  
  \bibitem{PhysRevB.105.174403}
  \bibinfo{author}{Gu, C.}, \bibinfo{author}{Yu, S.-L.} \& \bibinfo{author}{Li,
    J.-X.}
  \newblock \bibinfo{title}{Spin dynamics and continuum spectra of the honeycomb
    ${J}_{1}\text{\ensuremath{-}}{J}_{2}$ antiferromagnetic {Heisenberg} model}.
  \newblock \emph{\bibinfo{journal}{Phys. Rev. B}}
    \textbf{\bibinfo{volume}{105}}, \bibinfo{pages}{174403}
    (\bibinfo{year}{2022}).
  
  \end{thebibliography}
  
\clearpage
%%%%%%%%%%%%%%%%%%%%%%%%%%%%%%%%%%%%%%%%%%%%%%%%%%%%%%%%%%%%%%%%%%%%%%%
\title{Supplementary information for Quantum phase transitions  and composite  excitations of antiferromagnetic quantum spin trimer chains in a magnetic field}

\author{Jun-Qing Cheng}
%\thanks{These authors contributed equally to this work.}
\affiliation{State Key Laboratory of Optoelectronic Materials and Technologies, Center for Neutron Science and Technology, Guangdong Provincial Key
Laboratory of Magnetoelectric Physics and Devices, School of Physics, Sun Yat-Sen University, Guangzhou 510275, China}
\affiliation{School of Physical Sciences, Great Bay University, 523000, Dongguan, China, and
Great Bay Institute for Advanced Study, Dongguan 523000, China}

\author{Zhi-Yao Ning}

\affiliation{State Key Laboratory of Optoelectronic Materials and Technologies, Center for Neutron Science and Technology, Guangdong Provincial Key
Laboratory of Magnetoelectric Physics and Devices, School of Physics, Sun Yat-Sen University, Guangzhou 510275, China}


\author{Han-Qing Wu}
\email{wuhanq3@mail.sysu.edu.cn}
\affiliation{State Key Laboratory of Optoelectronic Materials and Technologies, Center for Neutron Science and Technology, Guangdong Provincial Key
Laboratory of Magnetoelectric Physics and Devices, School of Physics, Sun Yat-Sen University, Guangzhou 510275, China}


\author{Dao-Xin Yao}
\email{yaodaox@mail.sysu.edu.cn}
\affiliation{State Key Laboratory of Optoelectronic Materials and Technologies, Center for Neutron Science and Technology, Guangdong Provincial Key
Laboratory of Magnetoelectric Physics and Devices, School of Physics, Sun Yat-Sen University, Guangzhou 510275, China}



\maketitle


\begin{center}
  {\centering \bf Supplementary Note 1: ED results}
  \end{center}


  \begin{figure*}[t]
	\includegraphics[width=12cm]{Szz-g0.8-ED}
   \caption{\label{szz-ed} \textbf{ $\mathcal{S}^{zz}  (q,\omega) $  obtained from ED calculations for different phases.}  $\mathcal{S}^{zz}  (q,\omega) $ in (a) XY-I phase, (b) $1/3 $ magnetization  plateau phase, and (c) XY-II phase. All results are from the case where $g=0.8$ and $L=24$. The  color coding of $\mathcal{S}^{zz} (q,\omega)$ uses a piecewise function with  the boundary value $U_0=0.2$.}
\end{figure*}


\begin{figure*}[t]
   \includegraphics[width=16cm]{Sxx-g0.8-ED}
   \caption{\label{sxx-g0.8-ED} \textbf{$\mathcal{S}^{xx}  (q,\omega) $  obtained from ED calculations for different phases.}  $\mathcal{S}^{xx}  (q,\omega) $ in  (a) XY-I phase, (b) $1/3 $ magnetization  plateau phase, (c) XY-II phase, and (d) Ferromagnetic phase. All results are from the case where $g=0.8$ and $L=24$. The  color coding of $\mathcal{S}^{xx} (q,\omega)$ uses a piecewise function with the boundary value $U_0=0.2$.}
\end{figure*}
\begin{figure*}[t]
   \includegraphics[width=16cm]{SXX-g0.3-ED}
   \caption{\label{sxx-g0.3-ED}  \textbf{ $\mathcal{S}^{xx}  (q,\omega) $  obtained from CPT and DMRG-TDVP calculations for different phases with weak intertrimer interaction.}  $\mathcal{S}^{xx}  (q,\omega) $ in  (a) XY-I phase, (b) $1/3 $ magnetization  plateau phase, (c) XY-II phase, and (d)Ferromagnetic phase. All results are from the case where $g=0.3$ and  $L=24$. The  color coding of $\mathcal{S}^{xx} (q,\omega)$ uses a piecewise function with the boundary value $U_0=0.2$.}
\end{figure*}


  In addition to the DMRG-TDVP and CPT calculations, we have also applied the ED method to study the spin dynamics. The ED method is a fundamental and straightforward approach for calculating the eigenenergies and eigenstates for a spin models with small size, which plays a crucial role in analyzing  the quantum phases transitions and magnetic excitations in spin systems. In the main text,    the quantum critical points of Fig.2(a) and  all results in Fig.(6)  are obtained through the ED calculations. Initially,  we perform ED calculations to obtain reliable results for further investigation, as these calculations are computationally efficient and require minimal resources. When working with larger system sizes, we employ symmetries to block diagonalize the Hamiltonian and reduce computational time and memory usage. Nevertheless,   the  finite-size effects remain significant. To acquire more dependable insights regarding the thermodynamic limit, advanced numerical methods, such as quantum Monte Carlo, DMRG and CPT are employed. By comparing the results obtained from different methods,  we can establish more credible conclusions.  
  
  In  Supplementary Fig.~\ref{szz-ed}, the longitudinal spin excitations $\mathcal{S}^{zz}  (q,\omega) $  of the XY-I, $1/3 $ magnetization  plateau phase, and XY-II phases are present, which  in excellent agreement with the findings presented  in Fig.~{3} of the main text, especially concerning the incommensurate wave numbers at zero energy and the   characteristics of excitations spectra of three phases. Furthermore, Supplementary Fig.~\ref{sxx-g0.8-ED} and Fig.~\ref{sxx-g0.3-ED} display the results of $\mathcal{S}^{xx}  (q,\omega) $ for  different phases,  which are consistent with the results presented  in Fig.(4) and Fig.(5) of the main text, respectively.
  
  %%%%%%%%%%%%%%%%%%%%%%%%%%%%%%%%%%%%%%%%%%%%%%%%%%%%%%%%%%%%%%%%%%%%%%%%%%%%%%%%%%
  \begin{center}
	{\centering \bf Supplementary Note 2: Dispersion relations}
	\end{center}

  In the main text, the dispersion relations provide valuable insights into understanding the excitation mechanisms of diverse spin dynamics. When $g$ is small, the intermediate-energy and high-energy excitations are predominantly localized in the trimers. To confirm the nature of these quasiparticles in different phases, we propose a scheme to obtain their dispersion relations based on the imitation of complex ground states.  These dispersion relations are consistent  with the DMRG-TDVP and CPT results on the location and band widths of excitations spectra (see Fig.5(e) and Fig.7 of main text), indicating that our understanding of the excitation remains accurate, despite the utilization of a highly simplified approximation for the ground state in our calculations. 
  
  Here, we outline the main  procedures for deriving the dispersion relations.  Without magnetic field, the trimer chain can be described by an effective antiferromagnetic Heisenberg model with the effective interaction dependent on the intertrimer  interaction $J_{\rm{eff}}=4J_2/9$ \cite{cheng2022}. Due to the weak intertrimer interaction and doubly degenerate ground state, each trimer can be mapped onto  an effective spin $S=1/2$. Therefore, the assumption of  the ground-state wave function of spin chain being a product state of ground states of each trimer provides an effective way to simulate the internal trimer excitations. When a magnetic field is introduced,  as discussed in the main text, the quantum phase transition presents, resulting in gapless or gapped ground states in different phases. We have found that the assumption is still valid for   analyzing the excitations in the XY-I and $1/3 $ magnetization  plateau phases. 
  In the XY-I phase, we assume that the ground-state wave function of trimer chain is a product state of  $\left|0\right\rangle$ and $\left|1\right\rangle$, 
  \begin{equation}
	  \label{ground state}
	  \left| {\psi}\right\rangle_{\mathrm{g}}=\left| {0}\right\rangle_1 \left| {1}\right\rangle_2 \cdots \left| {0}\right\rangle_{N-1}  \left| {1}\right\rangle_{N},
  \end{equation}
  where $\left|0\right\rangle$ and $\left|1\right\rangle$ is the ground state and first excited state of a single trimer, as shown in Fig.(6) of main text. We neglect the interactions,  and instead deal with $2^(N-1)$ degenerate ground states, while enforcing the magnetic quantum numbers $\sum M_i = 0 $. Even if the  magnetic field leads to  an incommensurate ground state with a little magnetization, above roughly presupposition still works. It should be noted that we are targeting excitations above the $2^N$-fold degenerate ground-state manifold, rather than employing degenerate  perturbation theory. In our study,  we do not undertake  a formal perturbation expansion. Instead, we choose to construct intuitive variational states that encompass the internal excitations of a single trimer. 
  
  For the intermediate-energy excitations, we choose the $r$th trimer to be excited from $\left|0\right\rangle$ to $\left|3\right\rangle$ with $\Delta M =-1$ or  from $\left|1\right\rangle$ to $\left|2\right\rangle$ with $\Delta M =1$,   the excited wave function is then given by 
  
  \begin{equation}
	  \left| {\psi}\right\rangle_{\mathrm{e}}^r=\left| {0}\right\rangle_1 \left| {1}\right\rangle_2 \cdots  \left| {3}\right\rangle_r\cdots \left| {0}\right\rangle_{N-1} \left| {1}\right\rangle_{N}.
  \end{equation}
  To give this excitation a momentum, we perform a Fourier transformation on this excited state, resulting in 
  \begin{equation}
	  \left| {\psi }\right\rangle_{\mathrm{e}}^q=\frac{1}{\sqrt{N}} \sum_{r=1}^{N} \mathrm{e}^{-\mathrm{i} \emph{q} \emph{r}} \left| {\psi}\right\rangle_{\mathrm{e}}^r.
  \end{equation}
  Next, we can calculate the expectation values of the Hamiltonian in the ground state and the first excited trimer momentum state to obtain the dispersion relations corresponding to the intermediate-energy excitations in the reduced Hilbert space,
  \begin{eqnarray}
	  \epsilon(q)&=&\left\langle H \right\rangle_{\mathrm{e}} - \left\langle H \right\rangle_{\mathrm{g}} \nonumber \\
	  &=& ^{q}_{\mathrm{e}}\left\langle \psi\right|H \left| \psi\right\rangle^{q}_{\mathrm{e}} -^{q}_{\mathrm{g}}\left\langle \psi\right|H \left| \psi\right\rangle^{q}_{\mathrm{g}}.
  \end{eqnarray}
  Thus, the dispersion relations corresponding to the intermediate-energy doublon are given by,
  \begin{equation}
	  \epsilon_{\mathrm{D}}^{\mathrm{red}} (q)=\left\{
	  \begin{split}
  & -  \frac{1}{3} g \cos{q}+E_1 -E_0 + H_z - \frac{1}{9}g,\\
  & - \frac{2}{9} g \cos{q}+E_1 -E_0 + H_z,\\
  & - \frac{2}{9} g \cos{q}+E_1 -E_0 + H_z + \frac{2}{9}g,\\
  & - \frac{2}{9} g \cos{q}+E_1-E_0  + H_z + \frac{2}{9}g, 
	  \end{split}
   \right.
  \end{equation}
  which are independent of the length of the spin chain.
  By replacing $q$  with $3q$, we can obtain the unfolded dispersion relations in full Brillouin zone, as shown in Eq.(11) of main text.
  To visualize the calculation process, we plot Supplementary Fig.~\ref{dispersion-cal}, where it can be observed that only $4$ trimers are sufficient to obtain all the dispersion relations. 
  For the high-energy excitations, the $r$th trimer is excited 
  from $\left|0\right\rangle$ to $\left|4\right\rangle$ with $\Delta M =1$ or  from $\left|0\right\rangle$ to $\left|6\right\rangle$ with $\Delta M =-1$. Similar calculations can be performed to obtain the dispersion relations of the high-energy modes. However, the magnetic field splits the  high-energy spectra into two branches due to the diverse spin quantum numbers $\Delta M = \pm 1$.  These branches are  referred to as  the upper quarton and lower quarton (see Eq.(12) and Eq.(13) of main text), respectively. 
  
  In the   $1/3 $ magnetization  plateau phase, the gapped ground state simplifies  the calculations. As shown in  Supplementary Fig.~\ref{dispersion-cal}(b),  the  ground sate is constructed from product of polarized trimers (as the effective spins $S_{\rm{eff}}=1/2$).  The low-energy excitation originates from a flipped effective spin, similar to the formation of spin wave,  but in a reduced Hilbert space. Therefore, the  excitation from $\left|0\right\rangle$ to $\left|1\right\rangle$ with $\Delta M =-1$ is described by the reduced spin wave.
  
  
  Other excitations also arise from the internal trimer excitations, such as  $\left|0\right\rangle \rightarrow \left|3\right\rangle$ with $\Delta M =-1$, $\left|0\right\rangle \rightarrow \left|4\right\rangle$ with $\Delta M =1$, and $\left|0\right\rangle \rightarrow \left|6\right\rangle$ with $\Delta M =-1$. Each excitation  has only one  dispersion relation which consists well with the excitation spectrum. At the high-energy regime, the presence of a continuum with  weak weight may  originate from the fractional spinons, as discussed in our previous study \cite{cheng2022}. Furthermore, in Supplementary Fig.~\ref{large-g-dispersion}, we show that  the gap between ground state and first excited state provides effective protection for the excitations mentioned above. Even when the intertrimer interaction increases to $g=0.8$, the dispersion relations still effectively capture the internal trimer excitations. A higher value of the intertrimer interaction $g$ leads to the merging of these excitation spectra and gives rise to the formation of a continuum.
  
  
  
  
  
  
  
  \begin{figure*}[t]
	  \includegraphics[width=16cm]{dispersion}
	  \caption{\label{dispersion-cal} \textbf{Graphical representation of the calculation of the dispersion relation
	  $\epsilon(q)=\left\langle H \right\rangle_{\rm{e}} - \left\langle H \right\rangle_{\rm{g}}$.} (a) The  trimer eigenstates  are shown as darker blue (ground states)
	  and  red (excited states). With these states, the  excitations with $|\Delta M|=1$ originate from a trimer ground state
	  $\left|0\right\rangle^{1}_r$ on the trimer located at $r$ when excited to one of  $\left|1\right\rangle_r$, $\left|3\right\rangle_r$, $\left|4\right\rangle_r$, and $\left|6\right\rangle_r$ (and in the corresponding bra states we use the site index $l$ instead of $r$). The excitations are given momentum $q$, and the matrix elements contributing
	  to the dispersion relation are indicated. (b) The construction of ground state and excitation mechanism
  in the  $1/3 $ magnetization  plateau phase.}
  \end{figure*}
  
  \begin{figure*}[t]
	  \includegraphics[width=12cm]{Sxx-h1.2-dispersion.pdf}
	  \caption{ \label{large-g-dispersion}  \textbf{ $\mathcal{S}^{xx}  (q,\omega) $ in  $1/3 $ magnetization  plateau phase for different $g$.} All results are obtained by  DMRG-TDVP calculations  for $L=120$, and the  color coding of $\mathcal{S}^{xx} (q,\omega)$ uses a piecewise function with the boundary value $U_0=2$. The dispersion lines with colors and numbers are corresponding to the different localized excitations in a single trimer. (1)(2)(4) are the excitations from $\left| 0 \right\rangle \rightarrow \left| 1 \right\rangle$, $\left| 0 \right\rangle \rightarrow \left| 3 \right\rangle$ and $\left| 0 \right\rangle \rightarrow \left| 6 \right\rangle$ with $\Delta M =-1$, respectively. (3) is the  excitations from $\left| 0 \right\rangle \rightarrow \left| 4 \right\rangle$ with $\Delta M =1$.}
  \end{figure*}

% \begin{center}
% {\flushleft \bf Supplementary References}
% \end{center}
% \vspace{+1.5mm}
% %\bibliographystyle{apsrev4-1}
% \bibliographystyle{naturemag}
% %\bibliography{disorderfree}

% \begin{thebibliography}{10}

% \bibitem{cheng2022}
% \bibinfo{author}{Cheng, J.-Q.} \emph{et~al.}
% \newblock \bibinfo{title}{Fractional and composite excitations of
%   antiferromagnetic quantum spin trimer chains}.
% \newblock \emph{\bibinfo{journal}{npj Quantum Mater.}}
%   \textbf{\bibinfo{volume}{7}}, \bibinfo{pages}{1--11} (\bibinfo{year}{2022}).

% \end{thebibliography}










\end{document}
%
% ****** End of file apssamp.tex ******

